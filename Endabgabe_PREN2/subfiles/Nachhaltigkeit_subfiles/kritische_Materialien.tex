\documentclass[main.tex]{subfiles} % Subfile-Class

%==============================================================================%
%                                   Subfile                                    %
%==============================================================================%

\begin{document}

% Template

\subsection{Kritische Materialien und Alternativen}

Im Projekt wurden einige Materialien verwendet, die hinsichtlich Nachhaltigkeit 
als kritisch gelten. Die folgenden Beispiele zeigen typische Problemstellungen 
und mögliche Optimierungen:

\begin{itemize}
    \item \textbf{LiPo-Akku:} 
    Lithiumgewinnung ist wasserintensiv und erfolgt häufig unter fragwürdigen 
    Arbeitsbedingungen. Eine alternative Akkutechnologie wie NiMH wäre ökologisch 
    günstiger, jedoch schwerer und voluminöser.
    
    \item \textbf{Neodym-Magnete in Schrittmotoren:} Die Gewinnung seltener Erden 
    ist energie- und wasserintensiv und mit erheblichen Umweltfolgen verbunden. 
    Alternative Konzepte mit eisenbasierten Ferritmagneten könnten langfristig 
    eine nachhaltigere Lösung darstellen.
    
    \item \textbf{FR4-Leiterplatten:} Epoxidharz-Glasfaser-Substrate sind schwer 
    recycelbar. In frühen Entwicklungsphasen könnten wiederverwendbare 
    Entwicklungsboards oder modulare Stecksysteme den Materialverbrauch verringern.
\end{itemize}

\end{document}
