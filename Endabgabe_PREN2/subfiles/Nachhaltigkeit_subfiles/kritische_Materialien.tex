\documentclass[main.tex]{subfiles} % Subfile-Class

%==============================================================================%
%                                   Subfile                                    %
%==============================================================================%

\begin{document}

% Template

\subsection{Kritische Materialien und Alternativen}

Im Projekt wurden einige Materialien verwendet, die hinsichtlich Nachhaltigkeit 
als kritisch gelten. Die folgenden Beispiele zeigen typische Problemstellungen 
und mögliche Optimierungen:

\begin{itemize}
    \item \textbf{LiPo-Akku (Lithium-Polymer):} 
    Der enthaltene Rohstoff Lithium wird überwiegend im Lithiumdreieck Südamerikas 
    (Chile, Argentinien, Bolivien) durch Verdunstung aus Salzseen gewonnen. Dabei 
    werden pro Tonne Lithium bis zu \SI{2.2}{Millionen\liter} Wasser verdunstet, 
    was insbesondere in ariden Regionen zu drastischen Wasserknappheiten für 
    Landwirtschaft und Bevölkerung führt. Die soziale Problematik ergibt sich aus 
    dem Druck auf indigene Gemeinschaften und mangelnder Regulierung des Wasserrechts. 
    Als Alternative wären NiMH-Akkus denkbar, die zwar schwerer sind, jedoch keine 
    seltenen Rohstoffe wie Lithium enthalten.
    
    \item \textbf{Neodym-Magnete in Schrittmotoren:} 
    Neodym gehört zur Gruppe der seltenen Erden, deren Abbau meist in China erfolgt. 
    Die Gewinnung verursacht große Mengen an radioaktivem und schwermetallhaltigem Abraum.
    Auf eine Tonne Neodym kommen bis zu \SI{75}{Kubikmeter} giftiger Abfälle. Zudem ist 
    der Abbau stark energieintensiv, was sich durch den hohen Schmelzpunkt und aufwendige 
    Trennverfahren erklärt. Eine Alternative wären eisenbasierte Ferritmagnete, die zwar 
    ein geringeres Magnetfeld erzeugen, aber ökologisch deutlich unbedenklicher sind.
    
    \item \textbf{FR4-Leiterplatten (Epoxidharz mit Glasfaser):} 
    Diese Platinen sind aufgrund ihres Materialverbunds aus Harz und Glasfaser nur 
    schwer zu recyceln. Ihre Herstellung ist energieintensiv, und die Entsorgung erfolgt 
    meist durch thermische Verwertung. Eine stoffliche Trennung der Bestandteile ist in 
    der Praxis kaum wirtschaftlich möglich. In frühen Entwicklungsphasen könnten modulare 
    Breadboards oder wiederverwendbare Plattformen den Bedarf an neuen PCBAs deutlich 
    reduzieren.
\end{itemize}

\end{document}
