\documentclass[main.tex]{subfiles} % Subfile-Class

%==============================================================================%
%                                   Subfile                                    %
%==============================================================================%

\begin{document}

% Template

\subsection{Ökobilanz und Materialanalyse}

Das finale Fahrzeug hat ein Gesamtgewicht von \SI{1.84}{\kilo\gram}. Die folgende 
Tabelle zeigt die drei schwersten Materialkomponenten, deren jeweilige Masse und 
prozentualen Anteil am Gesamtgewicht:

\begin{table}[H]
\centering
\begin{tabular}{|l|c|c|}
\hline
\textbf{Materialkomponente} & \textbf{Masse (kg)} & \textbf{Anteil (\%)} \\
\hline
Elektronik (PCBAs, Sensorik, Motoren, Raspberry Pi) & 0.90 & 49 \\
PLA-Strukturteile & 0.55 & 30 \\
LiPo-Akku & 0.22 & 12 \\
\hline
\end{tabular}
\caption{Hauptmaterialien und ihre Masseanteile}
\end{table}

\subsubsection*{Rezyklierbarkeit und Abfallbehandlung}

\begin{itemize}
    \item \textbf{Elektronische Komponenten:} 
    Hochwertige Bauteile wie Motorentreiber, Schrittmotoren, Servomotoren, 
    Raspberry~Pi, Kamera und Lidar werden nach Projektende von Teammitgliedern in 
    privaten oder akademischen Projekten weiterverwendet. Die entwickelten PCBAs 
    (RaspberryHat, MotionController etc.) bieten viele Standard-I/Os und werden 
    zukünftigen PREN-Jahrgängen zur Verfügung gestellt. Sollte keine weitere Nutzung 
    erfolgen, erfolgt die Entsorgung fachgerecht über die Hochschule Luzern.
    
    \item \textbf{PLA:} 
    PLA wird häufig als biologisch abbaubar oder kompostierbar beworben. In der 
    Praxis ist jedoch eine industrielle Kompostierung erforderlich, die im normalen 
    Hausabfall nicht gewährleistet ist. Unser Ziel ist es daher, die PLA-Teile 
    industriell kompostieren zu lassen. Sollte dies nicht möglich sein, erfolgt 
    eine Entsorgung über den Restabfall.
    
    \item \textbf{LiPo-Akku:} 
    Der eingesetzte \SI{4}{S}~LiPo-Akku weist noch eine gute Restkapazität auf 
    und wird nach Projektende von einem Teammitglied in einem privaten Projekt 
    weiterverwendet. Eine Entsorgung ist daher nicht erforderlich. Sollte der Akku 
    im späteren Verlauf nicht mehr funktionstüchtig sein, wird er sicher und 
    normgerecht über eine spezialisierte Rücknahmestelle wie den Ökihof entsorgt. 
    Dabei wird insbesondere auf die schadstoffarme Behandlung und das Recycling 
    der Zellchemie geachtet.
\end{itemize}

\end{document}
