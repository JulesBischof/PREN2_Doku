\documentclass[main.tex]{subfiles} % Subfile-Class

%==============================================================================%
%                                   Subfile                                    %
%==============================================================================%

\begin{document}

% Template

\subsection{Ökobilanz und Materialanalyse}

Das finale Fahrzeug hat ein Gesamtgewicht von \SI{1.84}{\kilo\gram}. Die folgende 
Tabelle zeigt die drei schwersten Materialkomponenten, deren jeweilige Masse und 
prozentualen Anteil am Gesamtgewicht:

\begin{table}[H]
\centering
\begin{tabular}{|l|c|c|}
\hline
\textbf{Materialkomponente} & \textbf{Masse (kg)} & \textbf{Anteil (\%)} \\
\hline
Elektronik (PCBAs, Sensorik, Motoren, Raspberry Pi) & 0.90 & 49 \\
PLA-Strukturteile & 0.55 & 30 \\
LiPo-Akku & 0.22 & 12 \\
\hline
\end{tabular}
\caption{Hauptmaterialien und ihre Masseanteile}
\end{table}

\subsubsection*{Rezyklierbarkeit und Abfallbehandlung}

\begin{itemize}
    \item \textbf{Elektronische Komponenten:} 
    Hochwertige Bauteile wie Motorentreiber, Schrittmotoren, Servomotoren, 
    Raspberry~Pi, Kamera und Lidar werden nach Projektende von Teammitgliedern in 
    privaten oder akademischen Projekten weiterverwendet. Die entwickelten PCBAs 
    (RaspberryHat, MotionController etc.) bieten zahlreiche Standard-I/Os und 
    stehen künftigen PREN-Jahrgängen als universelle Hardwareplattform zur 
    Verfügung. Sollte sich mittelfristig kein Einsatzbereich ergeben, sorgt die 
    Hochschule Luzern für eine fachgerechte Entsorgung.
    
    \item \textbf{PLA:} 
    PLA wird häufig als biologisch abbaubar oder kompostierbar beworben. In der 
    Praxis ist dafür eine industrielle Kompostierung erforderlich, da die 
    vollständige Zersetzung unter natürlichen Bedingungen kaum realisierbar ist. 
    Unser Ziel ist es, sämtliche PLA-Bauteile gesammelt einer geeigneten 
    industriellen Kompostieranlage zuzuführen, um so eine möglichst umweltschonende 
    Materialbehandlung zu gewährleisten.
    
    \item \textbf{LiPo-Akku:} 
    Der eingesetzte \SI{4}{S}~LiPo-Akku weist noch eine gute Restkapazität auf und 
    wird nach Projektende fix von einem Teammitglied in einem privaten Projekt 
    weiterverwendet. Eine Entsorgung ist daher nicht erforderlich. Sollte der Akku 
    im späteren Verlauf nicht mehr funktionstüchtig sein, wird er sicher und 
    normgerecht über eine spezialisierte Rücknahmestelle wie den Ökihof entsorgt.
\end{itemize}

\end{document}
