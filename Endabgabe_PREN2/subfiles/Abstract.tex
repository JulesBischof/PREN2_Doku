\documentclass[main.tex]{subfiles} % Subfile-Class

%==============================================================================%
%                                   Subfile                                    %
%==============================================================================%

\begin{document}

\section*{Abstract}

Im Projektmodul \textbf{PREN 2} wurde das in \textbf{PREN 1} ausgearbeitete
Konzept eines autonomen Roboters erfolgreich in Hardware und Software
umgesetzt und unter realen Wettbewerbsbedingungen getestet.
Kern des Systems ist ein Raspberry-Pi-basiertes Steuergerät, das eine
endliche Zustandsmaschine zur Navigation auf einem weiss markierten
Knoten-Kanten-Netz ausführt. Die Bildverarbeitung der PiCamera2
liefert in Echtzeit die Richtung der befahrbaren Kanten sowie den
jeweiligen Knotenpunkt, während ein LIDAR Kurzstreckenhindernisse
detektiert. Ein Greedy-Algorithmus wählt fortlaufend jene Fahrtrichtung,
die die euklidische Distanz zum vor dem Start gewählten Zielknoten
(\texttt{A}, \texttt{B} oder \texttt{C}) am stärksten reduziert; dadurch
konnte auf komplexe globale Re-Planning-Strategien verzichtet werden.\\

Auch mechanisch wurde das System gezielt optimiert. Das Chassis besteht 
vollständig aus PLA und wurde im 3D-Druck gefertigt, um Gewicht und Bauraum 
zu minimieren. Zwei angetriebene Hinterräder mit eigens entwickelten Felgen 
und eine vordere Laufkugel ermöglichen hohe Wendigkeit. Die Greifeinheit 
ist modular aufgebaut, leicht konstruiert und greift Hindernisse mit zwei 
synchron bewegten Backen, die mit Anti-Rutsch-Band ausgekleidet sind. Ihre 
vertikale Bewegung erfolgt motorisiert entlang einer Gleitführung, wobei 
Endschalter die Endlagen erfassen. Die Sensorik wurde gezielt positioniert, 
um eine stabile Erkennung von Linie, Pylonen und Hindernissen zu 
gewährleisten.\\

Nachhaltigkeitsaspekte wie energieeffiziente Komponenten, ressourcenschonende
Materialwahl und der Bezug zu den Zielen für nachhaltige Entwicklung (SDGs) wurden
systematisch berücksichtigt.\\

Umfangreiche Feldversuche auf dem zur Verfügung gestellten Testparcours belegen,
dass das Gesamtsystem unter wechselnden Lichtverhältnissen zuverlässig
navigiert, Hindernisse sicher erkennt und das Ziel ohne manuelle Eingriffe
erreicht. Diese Ergebnisse stärken das Vertrauen des Teams, im anstehenden
Wettbewerb eine solide Leistung abzurufen und das Potenzial des entwickelten
Roboters voll auszuschöpfen.

\end{document}
