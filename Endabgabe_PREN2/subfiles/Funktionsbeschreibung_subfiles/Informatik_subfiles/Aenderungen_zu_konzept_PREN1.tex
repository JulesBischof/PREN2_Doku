\documentclass[main.tex]{subfiles} % Subfile-Class

%==============================================================================%
%                                   Subfile                                    %
%==============================================================================%

\begin{document}

\subsubsection{Änderungen zum Konzept \mbox{PREN 1}}

\paragraph{Überblick}
Das ursprüngliche Konzept aus PREN 1 sah einen DFS-basierten
Graphsuchalgorithmus mit sektionsabhängigen Heuristiken und umfangreicher
Sensorik (Gyroskop, Motor-Encoder, Lidar) vor.
Während der praktischen Umsetzung zeigte sich jedoch, dass mehrere Annahmen
dieses Konzepts nicht mit den realen Einschränkungen von Hardware, Zeit und
Feldbedingungen vereinbar waren.
Die wichtigsten Anpassungen sind unten zusammengefasst.

\paragraph{Algorithmische Änderungen}
\begin{itemize}
  \item \textbf{State-Machine statt rekursivem DFS}:
    Der gesamte Navigationsprozess wurde als endliche Zustandsmaschine
    implementiert.  Dadurch lassen sich Sensorereignisse deterministisch
    abhandeln, ohne tiefe Rekursion oder dynamische Speicherallokation.
  \item \textbf{Greedy-Zielannäherung}:
    Anstelle der sektionsabhängigen Gewichtung
    wählt der Planner jetzt jeweils den Kamerawinkel, der einen virtuellen
    Schritt mit minimaler Distanz zum Ziel ergibt.
    Die früheren Heuristikkonstanten
    (\emph{Section Boost}, \emph{Wrong Direction Penalty} usw.)
    wurden gestrichen.
  \item \textbf{Runtime-Graph nur für Logging}:
    \texttt{networkx} wird ausschliesslich verwendet, um befahrene
    Knotenpunkte und Kanten für Debug-Zwecke zu protokollieren.
    Globales Re-Planning ist entfallen.
  \item \textbf{Vereinfachter Umgang mit Schranken}:
    Schranken werden wie reguläre Kanten behandelt;
    eine dynamische Neubewertung entfällt.
\end{itemize}

\paragraph{Softwarearchitektur}
\begin{itemize}
  \item \textbf{High-Level-Controller-Thread}:
    Sensorfusion, Planner und UART-Kommunikation laufen nun in einem
    separaten Entscheidungsthread.  Dadurch bleibt der Hauptthread frei
    für Start/Stop-Logik und Ausnahmebehandlung.
  \item \textbf{Schiebeschalter statt Software-Konfiguration}:
    Das Ziel (\texttt{A}/\texttt{B}/\texttt{C}) wird hardwareseitig vor
    dem Start gewählt; eine Laufzeitkonfiguration entfällt.
\end{itemize}

\paragraph{Auswirkungen}
Die genannten Anpassungen verkleinern den Rechen- und
Implementierungsaufwand deutlich und erhöhen die Robustheit in der
Turnierumgebung.  Gleichzeitig verzichtet das System bewusst auf global
optimale Pfade und nimmt längere Strecken in Kauf, falls kurzfristig eine
Kante entfernt oder gesperrt wird.

\end{document}
