\documentclass[main.tex]{subfiles} % Subfile-Class

%==============================================================================%
%                                   Subfile                                    %
%==============================================================================%

\begin{document}

%-------------------------------------------------------------
\subsubsection{UART-Protokoll - Kommunikation zwischen den Teilsystemen}
%-------------------------------------------------------------

\subsubsection*{Einleitung}
Dieses Kapitel beschreibt ein festes 64-Bit-UART-Protokoll, das für
das autonome Fahrzeug von Team 10 (PREN2) entwickelt wurde. Die
wichtigsten Anforderungen sind
\begin{itemize}
  \item geringer Overhead,
  \item hohe Robustheit durch CRC-Fehlererkennung,
  \item konstante Rahmenlänge (zwei 32-Bit-Lesezyklen pro Frame),
  \item Übertragung von bis zu 50 Bit Nutz- bzw.\ Reservedaten.
\end{itemize}

\bigskip
Jeder 64-Bit-Frame ist folgendermassen aufgebaut:
\[
  \underbrace{\text{Addr}}_{2\;\text{Bit}}
  \;\|\;
  \underbrace{\text{Cmd}}_{4\;\text{Bit}}
  \;\|\;
  \underbrace{\text{Param/Res}}_{50\;\text{Bit}}
  \;\|\;
  \underbrace{\text{CRC-8}}_{8\;\text{Bit}}
\]

%-------------------------------------------------------------
\subsubsection*{Rahmendefinition}
Ein UART-Frame umfasst exakt 64 Bit. Die Felder lauten:

\begin{itemize}
  \item \textbf{Adresse (2 Bit)}:
    \[
      \text{Addr} \in \{0b00,\,0b01,\,0b11\}
    \]
    Beispielzuordnung:
    \begin{align*}
      0b00 &\;\rightarrow\; \text{Raspberry Hat},\\
      0b01 &\;\rightarrow\; \text{Motion-Controller},\\
      0b11 &\;\rightarrow\; \text{Grip-Controller}.
    \end{align*}

  \item \textbf{Befehl (4 Bit)}:
    \[
      \text{Cmd} \in \{0x0,\dots,0xF\}
    \]
    Die aktuell verwendeten Codes sind in Tab.~\ref{tab:cmds} aufgeführt.

  \item \textbf{Parameter/Reserve (50 Bit)}:
    \[
      \underbrace{p_{49}\,p_{48}\,\dots\,p_{1}\,p_{0}}_{50\;\text{Bit}}
    \]
    Dieses Feld kodiert je nach Befehl unterschiedliche Datentypen
    (vgl.\ Tab.~\ref{tab:types}); ungenutzte Bits werden reserviert
    oder mit 0 gefüllt.

  \item \textbf{CRC-8 (8 Bit)}:
    \[
      \underbrace{c_{7}\,c_{6}\,\dots\,c_{1}\,c_{0}}_{8\;\text{Bit}}
    \]
    Standard-CRC-8 über die oberen 56 Bit.
\end{itemize}

%-------------------------------------------------------------
\newpage
\subsubsection*{Befehlsreferenz}
\begin{table}[h!]
  \centering
  \begin{tabular}{>{\ttfamily}c >{\ttfamily}c l}
    \toprule
    \textnormal{Hex} & \textnormal{Binär} & \textnormal{Befehl / Parameter} \\
    \midrule
    0x0 & 0b0000 & Move (\texttt{uint16}: Strecke [cm], 0 = unendlich) \\
    0x1 & 0b0001 & Reverse (\texttt{int16}: Strecke [cm], 0 = unendlich) \\
    0x2 & 0b0010 & Turn (\texttt{int16}: Winkel [°], +links / -rechts) \\
    0x3 & 0b0100 & Stop (keine Parameter) \\
    0x4 & 0b0110 & Info (\texttt{uint8} Flags) \\
    0x5 & 0b0111 & Ping (\texttt{uint8} ID) \\
    0x6 & 0b1000 & Pong (\texttt{uint8} ID) \\
    0x7 & 0b1001 & Error (\texttt{uint16} Code) \\
    0x8 & 0b1010 & Poll (\texttt{uint8} Abfrage) \\
    0x9 & 0b1011 & Response (\texttt{uint8} Typ, \texttt{uint16} Daten) \\
    0xA & 0b1100 & Grip (keine Parameter) \\
    0xB & 0b1101 & Release (keine Parameter) \\
    \bottomrule
  \end{tabular}
  \caption{Befehlscodes in Hex und Binär}
  \label{tab:cmds}
\end{table}

%-------------------------------------------------------------
\subsection*{Datentypreferenz}
\begin{table}[h!]
  \centering
  \begin{tabular}{l l}
    \toprule
    \textbf{Typ} & \textbf{Wertebereich / Beschreibung} \\
    \midrule
    \texttt{int16} & -32 768 … 32 767 \\
    \texttt{int8}  & -128 … 127 \\
    \texttt{uint16} & 0 … 65 535 \\
    \texttt{uint8}  & 0 … 255 \\
    \texttt{fp16}   & IEEE-754 Half Precision \\
    \texttt{fp8}    & 8-Bit-Minifloat \\
    \bottomrule
  \end{tabular}
  \caption{Mögliche Datentypen für das Parameterfeld}
  \label{tab:types}
\end{table}

%-------------------------------------------------------------
\newpage
\subsubsection*{CRC-8-Berechnung}

\subsubsection*{Polynom}
Verwendet wird das primitive Polynom
\[
  x^{8} + x^{2} + x + 1 \;(\texttt{0x07}).
\]

\subsection*{Motivation}
Das Polynom ist primitiv über \(\mathrm{GF}(2)\) und bietet dadurch:
\begin{itemize}
  \item \textbf{Erkennung beliebiger Einzelbitfehler}.
  \item \textbf{Hohe Erkennungsrate für Doppelbitfehler}.
  \item \textbf{Sichere Erkennung aller Burstfehler bis 8 Bit Länge}.
\end{itemize}
Bei einem nur 64 Bit langen Frame liegt die theoretische
Restfehlerwahrscheinlichkeit bei maximal \(1/2^{8}=0{,}39\,\%\).

\subsection*{Bitweise Berechnung}
\begin{enumerate}
  \item \(\text{crc} \gets 0x00\)
  \item 7 Bytes (56 Bit) des Frames sequentiell verarbeiten.
  \item Für jedes Byte \texttt{b}:
    \[
      \text{crc} \gets \text{crc} \oplus b
    \]
    Danach für 8 Bit:
    \[
      \text{crc} \gets
      \begin{cases}
        (\text{crc}\ll1) \oplus 0x07, & \text{falls Bit7 gesetzt} \\
        (\text{crc}\ll1),            & \text{sonst}
      \end{cases}
    \]
  \item Ergebnis (\texttt{crc}) als letzte 8 Bit anfügen.
\end{enumerate}

%-------------------------------------------------------------
\subsection*{Sende- und Empfangsablauf}

\subsection*{Sendesequenz}
\begin{enumerate}
  \item 56-Bit-Nutzteil bilden:
    \(\text{raw} = (Addr\ll62) | (Cmd\ll58) | (Param\ll8)\)
  \item CRC-8 über \text{raw} berechnen.
  \item CRC als niederwertigste 8 Bit anhängen.
  \item 64-Bit-Frame via UART senden.
\end{enumerate}

\subsection*{Empfangssequenz}
\begin{enumerate}
  \item 64 Bit einlesen.
  \item Obere 56 Bit (\texttt{Addr + Cmd + Param}) und CRC trennen.
  \item CRC-8 neu berechnen.
  \item Vergleich:
    \(\text{crc}_{\text{calc}} \stackrel{?}{=} \text{crc}_{\text{rx}}\)
    Ungleich → Frame verwerfen.
  \item Gleich → Frame akzeptieren und gemäss Protokoll auswerten.
\end{enumerate}

\end{document}
