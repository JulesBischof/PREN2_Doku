\documentclass[main.tex]{subfiles} % Subfile-Class

%==============================================================================%
%                                   Subfile                                    %
%==============================================================================%

\begin{document}

\subsubsection{Wegfind-Algorithmus}

\paragraph{Grundidee}
Der \texttt{PathPlanner} läuft als endliche Zustandsmaschine auf dem
Raspberry Pi.
Er verarbeitet kontinuierlich Kamera-, Lidar- und Odometriedaten und erzeugt
daraus UART-Kommandos für den Motion-Controller.
Das Fahrzeug folgt weissem Klebeband:
\emph{Knotenpunkte} sind die weissen Kreise,
\emph{Kanten} die Bandstücke dazwischen.
Ein Knotenpunkt kann durch einen Pylon blockiert sein,
Kanten können durch Barrieren unpassierbar oder von den Organisatoren entfernt
werden.
Das Ziel ist einer der drei markierten Knotenpunkte \texttt{A}, \texttt{B} oder
\texttt{C} und wird vor dem Start per Hardware-Schiebeschalter gewählt.

\paragraph{Laufzeit-Graph (\texttt{nxgraph})}
\vspace{-0.3\baselineskip}
\begin{itemize}
  \item Implementiert als \texttt{networkx.Graph}.
  \item Beim Einfahren in einen Knotenpunkt wird dessen Koordinate (Odometrie)
    mit allen bekannten Knotenpunkten verglichen
    (Toleranz $\approx200\,\mathrm{mm}$).
    Passt keiner, erhält der neue Punkt die ID \texttt{n1}, \texttt{n2}, ….
  \item Kanten zum Vorgängerknotenpunkt werden mit der real gefahrenen Distanz
    $d$ als Kantengewicht eingetragen.
  \item Attribute \texttt{traversable=false} markieren blockierte Knoten-
    oder Kantenpunkte; diese werden bei der Richtungswahl ignoriert.
  \item Der Graph dient primär als Laufzeitlog und für Debugging;
    er wird noch nicht für globales Re-Planning verwendet.
\end{itemize}

\paragraph{Zustandsmaschine}
\vspace{-0.3\baselineskip}
\begin{description}
  \item[\textbf{BEGIN}]
    Initialisierung, sofort \texttt{MOVE(0)} $\rightarrow$
    \textbf{TRAVELING\_EDGE}.

  \item[\textbf{TRAVELING\_EDGE}]
    Geradeausfahrt auf aktueller Kante.
    Erste Kamerawinkel $\alpha_i$ empfangen $\rightarrow$ Stopp,
    \textbf{ARRIVED\_AT\_NODE}.

  \item[\textbf{ARRIVED\_AT\_NODE}]
  \begin{itemize}\itemsep0pt
      \item Positionsupdate; Einfügen oder Matching im \texttt{nxgraph}.
      \item Kante zum Vorgänger mit Gewicht $d$ eintragen.
      \item Zielknoten erkannt ($<$\,200 mm oder Kameraflag)
        $\rightarrow$ \textbf{GOAL\_REACHED}.
      \item Sonst Push auf Stack $\rightarrow$ \textbf{DECIDING\_NEXT\_ANGLE}.
    \end{itemize}

  \item[\textbf{DECIDING\_NEXT\_ANGLE}]
  \begin{itemize}\itemsep0pt
      \item Greedy-Evaluation aller Kamerawinkel~$\alpha_i$:
        simulierter Schritt $0.5\,\mathrm{m}$, Distanz zum Ziel minimal.
      \item Keine Winkel $\rightarrow$ \textbf{BLOCKED}.
      \item Befehl: \texttt{TURN}$(\pm180-\alpha_{\text{best}})\cdot10$
        $\rightarrow$ \textbf{CHECK\_NEXT\_ANGLE}.
    \end{itemize}

  \item[\textbf{CHECK\_NEXT\_ANGLE}]
    \texttt{MOVE(0)} starten, zurück zu \textbf{TRAVELING\_EDGE}.

  \item[\textbf{GOAL\_REACHED}]
    Endzustand, \texttt{STOP}.

  \item[\textbf{BLOCKED}]
    Notstopp, \texttt{STOP}.
\end{description}

\paragraph{Greedy-Winkelauswahl}
Für jeden Kamerawinkel $\alpha_i$ wird ein Schrittpunkt bestimmt:
\[
  p_i = p_{\text{aktuell}} +
  \mathrm{Rot}_{\alpha_i+\varphi}\!\bigl(0,\;0.5\,\mathrm{m}\bigr),
  \qquad
  d_i = \lVert p_i - p_{\text{ziel}}\rVert_2.
\]
Der Winkel mit minimalem $d_i$ wird gewählt,
danach Orientierung
$\varphi \leftarrow (\varphi + \alpha_{\text{best}}) \bmod 360^{\circ}$.

\paragraph{Positionsupdate}
Nach jeder gemeldeten Strecke $d$:
\[
  p \leftarrow p + d\,
  \bigl(\cos\varphi,\;\sin\varphi\bigr).
\]

\paragraph{Stärken und Limiten}
\vspace{-0.3\baselineskip}
\begin{itemize}
  \item Sehr geringe CPU-Last, ideal für Echtzeit auf dem RPi.
  \item Laufzeit-Graph visualisiert tatsächlich befahrene Knoten- und
    Kantenpunkte.
  \item Kein globales Re-Planning: Fällt eine Kante erst spät weg,
    kann das Fahrzeug in einer Sackgasse landen.
\end{itemize}

\end{document}
