\documentclass[main.tex]{subfiles} % Subfile-Class

%==============================================================================%
%                                   Subfile                                    %
%==============================================================================%

\begin{document}

% Template

\subsection{Mechanik}

Das Mechanik-Konzept bildet das Grundgerüst und die funktionale Grundlage des 
autonomen Roboters. Im Fokus stehen die optimale Integration der Anforderungen 
bezüglich Gewicht, Kosten und Zuverlässigkeit sowie die effiziente mechanische 
Umsetzung der Funktionen \enquote{Fahren} und \enquote{Greifen}. Dabei wurde auf dem Konzept 
von PREN 1 aufgebaut und anhand von Erfahrungen aus den ersten Prototypen 
Verbesserungen eingebracht. Im folgenden Kapitel werden die einzelnen 
mechanischen Baugruppen detailliert beschrieben, beginnend mit der 
Chassiskonstruktion über die Greifeinheit bis hin zu den wichtigsten 
Anpassungen im Vergleich zu dem Konzept aus PREN 1.

\subfile{./Mechanik_subfiles/Chassis.tex}
\newpage

\subfile{./Mechanik_subfiles/Greifeinheit.tex}
\newpage

\subfile{./Mechanik_subfiles/Aenderungen_zu_konzept_PREN1.tex}
\newpage

\end{document}
