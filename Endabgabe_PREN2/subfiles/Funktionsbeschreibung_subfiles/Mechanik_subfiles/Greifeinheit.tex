\documentclass[main.tex]{subfiles} % Subfile-Class

%==============================================================================%
%                                   Subfile                                    %
%==============================================================================%

\begin{document}

% Template

\subsubsection{Greifeinheit}

\subsubsection*{Anforderungen}

Die Greifkraft muss so ausgelegt sein, dass das Hindernis zuverlässig und sicher gegriffen werden kann.
Dabei sind insbesondere die Haftreibung zwischen Greiffläche und Objekt sowie die notwendige Anpresskraft
zu berücksichtigen, um ein Abrutschen während des Transports zu vermeiden.
Die Höhenverstellung des Greifers muss auf das Gesamtgewicht der zu bewegenden Last abgestimmt sein.
Dazu zählen das Gewicht des Hindernisses, des Greifers selbst sowie der mitgeführten Elektronik.
Für den gesamten Bewegungsablauf des Greifers ist eine hohe Wiederholgenauigkeit erforderlich.
Diese stellt sicher, dass das Hindernis stets innerhalb des geforderten Toleranzbereichs platziert
wird und die Funktionalität des Gesamtsystems nicht beeinträchtigt wird.
Das Gewicht der Greifereinheit ist möglichst gering zu halten, um das Gesamtgewicht des Fahrzeugs zu minimieren.
Eine reduzierte Masse ist insbesondere bei den bewegten Teilen des Greifers von Bedeutung,
da sie direkten Einfluss auf die Auswahl der Antriebskomponenten, die erreichbare Geschwindigkeit
sowie die Energieeffizienz des Systems hat.

\subsubsection*{Konstruktive Umsetzung}

Die Abbildung~\ref{fig:Greifereinheit} zeigt das realisierte Greiferkonzept des Roboters.
Dieses wurde auf Basis der Erkenntnisse aus PREN1 weiterentwickelt und in PREN2 gezielt optimiert.
Das Greiferkonzept wurde überwiegend aus PLA im 3D-Druckverfahren gefertigt,
was eine schnelle und flexible Umsetzung sowie einfache Anpassungen in mehreren Iterationen ermöglichte.
Im folgenden Abschnitt wird die Funktionsweise des Greifers im Detail erläutert.

--------------------Bild Greifeinheit

\newpage

Der Greifer besitzt zwei gegenüberliegende Backen, die sich parallel zueinander bewegen,
um das Hindernis sicher zu greifen. Die Bewegung wird durch einen Servomotor gesteuert.
Ein Hebel, angetrieben vom Servomotor, verschiebt eine der beiden Backen horizontal.
Diese Backe ist mit einer Zahnstange versehen, die ein innenliegendes Zahnrad antreibt.
Das Zahnrad überträgt die Bewegung auf die gegenüberliegende Backe, wodurch sich beide
synchron, jedoch in entgegengesetzten Richtungen bewegen.

--------------------Bild innenliegendes Zahnrad

Auch die Höhenverstellung erfolgt
über einen Servomotor, der den gesamten Greifer entlang einer Gleitführung von IGUS vertikal bewegt.
Das für das Greifen erforderliche Drehmoment beträgt 0.235 $ \text{Nm}$,
für die Höhenverstellung 0.2 $\text{Nm}$. Diese Werte wurden mit den folgenden Formeln berechnet:

\[
    M_{Greifer} = \frac{m \cdot g}{\mu_{\text{hr}}} \cdot Hebel \cdot Sicherheit
\]

\[
    M_{Hoehe} = (m_{Hindernis} + m_{Elektronik} + m_{Greifer}) \cdot Hebel \cdot Sicherheit
\]

\newpage

Die Abbildung~\ref{fig:Greiferablauf} zeigt den Ablauf des Greifvorgangs in mehreren Bildern.
Die Bewegung des Motors werden mit einem gelben Pfeil dargestellt:

--------------------Bild

\paragraph{Positionierung der Greifeinheit}
Von Position 1 zu 2 dreht der linke Servomotor die Greifeinheit entlang einer Gleitführung
nach unten, nachdem ein Hindernis erkannt wurde.

\paragraph{Greifen des Hindernisses}

Von Position 2 zu 3 bewegt der rechte Servomotor die Backen horizontal, wodurch sich
beide Backen parallel schliessen und das Hindernis greifen.

\paragraph{Anheben des Hindernisses}

Von Position 3 zu 4 dreht der linke Servomotor die Greifeinheit leicht nach oben, sodass
das Hindernis vom Boden angehoben wird. Dadurch wird ein sicherer Transport des Hindernisses gewährleistet.



\newpage

\subsubsection{Sensorikplatzierung}

In diesem Abschnitt wird die Positionierung der Sensoren erläutert.

--------------------Bild

\paragraph{Raspberry Pi Kamera – Knotenpunkt}

Die Raspberry Pi Kamera ist in einer Höhe von \textbf{XXXXX mm} und in einem Winkel von \textbf{XXXXX°} montiert.
In dieser Position ist sie in der Lage, Objekte mit einem Durchmesser von etwa \textbf{XXXXX mm} auf
dem Boden zuverlässig zu erfassen. Dadurch wird sichergestellt, dass sowohl die \textbf{XXXXX mm} grossen
Knotenpunkte als auch deren Abgänge erkannt werden.

\paragraph{Ultraschallsensor und Lichtschranke – Hinderniserkennung}

Der Ultraschallsensor ist in einer Höhe von \textbf{XXXXX mm} angebracht und dient der frühzeitigen
Erkennung von Hindernissen, um Kollisionen mit dem Roboter zu vermeiden. Zusätzlich gewährleistet er,
dass der Roboter präzise an der vorgesehenen Position stoppt, um das Hindernis anzuheben. Die genaue
Funktionsweise wird im Kapitel~\ref{appendix:Abstandssensoren_Kapitel} detaillierter beschrieben.

\paragraph{LiDAR – Pylonerkennung}

Der LiDAR-Sensor ist in einer Höhe von \textbf{XXXXX mm} montiert und ermöglicht die Detektion von
Pylonen in einer Entfernung von bis zu \textbf{XXXXX Metern}. Die gewählte Montagehöhe stellt sicher,
dass innerhalb dieser Reichweite keine unerwünschten Streuungen auftreten, die zu Fehlmessungen führen
könnten. Eine ausführliche Beschreibung der Funktionsweise findet sich ebenfalls im
Kapitel~\ref{appendix:Abstandssensoren_Kapitel}.

\paragraph{Endschalter – Höhenverstellung des Greifers}

Die Endschalter sind an den Endpositionen der Höhenverstellung des Greifers angebracht.
Sie dienen der Positionsrückmeldung und verhindern ein Überdrehen des Servomotors. Dadurch wird
sichergestellt, dass der Greifer stets die korrekte Höhe erreicht und mechanische Beschädigungen vermieden werden.


\end{document}
