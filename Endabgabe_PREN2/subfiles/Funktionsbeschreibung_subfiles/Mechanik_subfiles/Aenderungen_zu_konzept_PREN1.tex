\documentclass[main.tex]{subfiles} % Subfile-Class

%==============================================================================%
%                                   Subfile                                    %
%==============================================================================%

\begin{document}

% Template

\subsubsection{Optimierungen gegenüber PREN 1}

Gegenüber dem Konzept aus PREN 1 erfolgte eine grundlegende Materialumstellung 
für das Chassis von MDF auf PLA. Zudem wurde die Herstellung mittels Lasercutter 
verworfen und vollständig auf das 3D-Druckverfahren umgestellt. Auch das 
Montagesystem der Elektronikkomponenten wurde angepasst, anstelle der Etagenlösung 
kamen einfache Distanzbolzen zum Einsatz (vgl. Kapitel~\ref{sec:Chassis}).

Im Fahrwerksbereich wurden die Originalfelgen der Antriebsräder durch 
eigenentwickelte Felgen aus dem 3D-Drucker ersetzt, um die Anbindung an die 
Motorwellen zuverlässig zu gestalten und auf problematische Adapterteile zu 
verzichten. Diese Optimierungen dienten vor allem der Gewichtsreduktion und 
verbesserten Zuverlässigkeit der mechanischen Konstruktion.

\end{document}
