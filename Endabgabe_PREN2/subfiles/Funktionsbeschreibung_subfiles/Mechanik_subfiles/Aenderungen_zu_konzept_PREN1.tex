\documentclass[main.tex]{subfiles} % Subfile-Class

%==============================================================================%
%                                   Subfile                                    %
%==============================================================================%

\begin{document}

% Template

\subsubsection{Optimierungen gegenüber PREN 1}

Im Vergleich zum Konzept aus PREN1 wurden im Rahmen von PREN2 umfangreiche Optimierungen an Mechanik, 
Materialwahl und Fertigungstechniken vorgenommen.

Eine grundlegende Änderung betraf das Chassis: Anstelle von MDF kam nun PLA als Hauptmaterial zum Einsatz. 
Gleichzeitig wurde die Herstellung vollständig auf das 3D-Druckverfahren umgestellt.
Auch das Montagesystem der Elektronikkomponenten wurde vereinfacht, anstelle der Etagenlösung 
kamen einfache Distanzbolzen zum Einsatz (vgl. Kapitel~\ref{sec:Chassis}).

Im Bereich des Fahrwerks wurden die Originalfelgen der Antriebsräder durch eigenentwickelte 
Felgen aus dem 3D-Drucker ersetzt. Dies ermöglichte eine zuverlässigere Verbindung mit den Motorwellen und 
machte problematische Adapterteile überflüssig. Diese Optimierungen dienten vor allem der Gewichtsreduktion und 
verbesserten Zuverlässigkeit der mechanischen Konstruktion.

Auch der Greifer wurde im Zuge von PREN2 gezielt weiterentwickelt. Zur Gewichtseinsparung wurden 
gezielt Materialaussparungen in der Struktur eingeführt. Die Toleranzen der Greiferbacken wurden 
in mehreren Iterationen angepasst, bis eine reibungsarme und präzise Gleitbewegung erreicht wurde.

Zur Erhöhung der Haftreibung auf den Greifflächen wurde ein Anti-Rutsch-Band angebracht. 
Zwischen dem Band und den Greiferbacken wurde ein leicht kompressibles, doppelseitiges Klebeband eingesetzt. 
Diese Kombination verbesserte nicht nur die Griffigkeit beim Aufnehmen des Hindernisses, sondern 
erleichterte auch die Regelung des Greifvorgangs durch den leicht nachgebenden Kontakt.

\end{document}
