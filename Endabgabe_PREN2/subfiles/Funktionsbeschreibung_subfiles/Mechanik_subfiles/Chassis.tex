\documentclass[main.tex]{subfiles} % Subfile-Class

%==============================================================================%
%                                   Subfile                                    %
%==============================================================================%

\begin{document}

% Template

\subsubsection{Chassis}

Dieser Abschnitt beschreibt die Evaluierung des Fahrwerks sowie die
Konstruktion des Chassis. Dabei werden die Anforderungen an Gewicht, Kosten und
Funktionalität sowie die konstruktiven Entscheidungen im Detail erläutert.

\subsubsection*{Anforderungen}

\textbf{Gewicht:} \newline
Das Gewicht stellt bei allen Baugruppen einen kritischen Faktor dar. Da
leistungsstarke Schrittmotoren und der Akku bereits einen Grossteil des
Gewichtsbudgets beanspruchen, verbleibt für das Chassis lediglich ein Budget von
200 Gramm. Dies erfordert eine leichte, aber dennoch stabile Konstruktion. Die
Gewichtsoptimierung wurde bereits in der frühen Planungsphase berücksichtigt, um
sicherzustellen, dass das Fahrzeug die Anforderungen hinsichtlich Stabilität und
Belastbarkeit erfüllt.

\textbf{Kosten:} \newline
Um die Kosten niedrig zu halten, soll auf die frei verfügbaren Ressourcen der
Hochschule Luzern zurückgegriffen werden. Das Team kann 25 Stunden Druckzeit am
HSLU-T\&A-Drucker sowie 1 Stunde Maschinenlaufzeit des Lasersystems kostenfrei
nutzen. Diese Ressourcen werden optimal ausgeschöpft, um den Einsatz von
Einkaufsteilen zu minimieren. Ziel ist es, das verfügbare Budget vor allem für
funktionskritische Baugruppen wie den Greifmechanismus und die Antriebssteuerung
einzusetzen.

\subsubsection*{Chassiskonstruktion}

Das Chassis basiert auf einer Grundplatte aus PLA, die im 3D-Druckverfahren hergestellt wurde. 
Im Gegensatz zur vorherigen MDF-Variante ist PLA zwar etwas schwerer, bietet jedoch den Vorteil, 
dass komplexe Strukturen und geometrische Formen direkt im Druckprozess integriert werden können. 
Dadurch konnten sämtliche Aussparungen und Montagelöcher bereits beim Drucken berücksichtigt werden, 
um gezielt Gewicht einzusparen und die Stabilität dennoch zu gewährleisten. 
Die Fertigung der Grundplatte erfolgte dabei im Heimdruck mit einem handelsüblichen 3D-Drucker. 
Abbildung~\ref{fig:Grundplatte} zeigt das Design der Grundplatte des finalen Fahrzeugs.

----Bild

Abbildung~\ref{fig:Chassis_komplett} zeigt das Chassis mit allen Anbauteilen, die im 
3D-Druckverfahren gefertigt wurden. Im Vergleich zu früheren Versionen ist das Design nun deutlich 
kompakter ausgelegt, da bestimmte Komponenten in dieser Phase nicht benötigt werden. Anstelle einer 
Etagenbauweise wird vermehrt mit Distanzbolzen gearbeitet, um Bauteile auf einfache Weise zu 
befestigen und dennoch eine klare Struktur zu behalten.

---Gewichtoptimierungs text


\subsubsection*{Fahrwerk}

Das Fahrwerk basiert auf einem Design mit zwei angetriebenen Rädern und einer
Laufkugel als dritten Auflagepunkt, um Stabilität zu gewährleisten. Diese
Lösung minimiert die Anzahl der Einkaufsteile und damit die Kosten. Zudem
bietet sie eine hohe Wendigkeit des Fahrzeugs. Abbildung~\ref{fig:Fahrwerk}
zeigt die Umsetzung mit den beiden Hinterrädern und der vorderen Laufkugel. Die
Platzierung der angetriebenen Räder an der Hinterachse ermöglicht präzise
Kurskorrekturen, da das Fahrzeug durch kleine Bewegungen der Hinterräder seine
Ausrichtung leicht ändern kann. Dies erlaubt es dem Fahrzeug, auf Knotenpunkten
effizient zu drehen und verschiedene Wegabzweigungen zu überprüfen.


Die Radgrösse wurde auf 80 mm festgelegt, wodurch eine maximale Geschwindigkeit
von 1,676~m/s erreicht wird. Die Berechnung erfolgt nach folgender Formel:

\[ v_{max} = n_{Motor} \cdot d_{Rad} \cdot \pi = 1.676 \, \frac{\text{m}}{\text{s}} \]

Die Räder vom Typ FIT0500 von DFRobot wurden im Rahmen einer Sammelbestellung der 
Hochschule Luzern beschafft, um die Umweltbelastung durch Einzelversendungen 
zu reduzieren. Die Felgen hingegen wurden im 3D-Druckverfahren gefertigt, um 
sie direkt an der Motorenwelle befestigen zu können. Dadurch konnte auf einen 
zusätzlichen Radadapter verzichtet werden, was den Aufbau vereinfacht.

\end{document}
