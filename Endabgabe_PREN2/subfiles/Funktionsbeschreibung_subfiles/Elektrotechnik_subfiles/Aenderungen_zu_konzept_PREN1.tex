\documentclass[main.tex]{subfiles} % Subfile-Class

%==============================================================================%
%                                   Subfile                                    %
%==============================================================================%

\begin{document}

% Template

\subsubsection{Änderungen zu Konzpet PREN 1}

\subsubsection*{Sensorik}

Auf eine redundante Abfrage der Hindernisposition in Form einer zusätzlichen
Lichtschranke neben dem Ultraschallsensor ist verzichtet worden. Mittels
Systemtheoretischer betrachtungen und ein wenig Statistik ist die
Messungenauigkeit auf ein Mass reduziert, mit welchem der Roboter auch alleine
auf Basis dieser Messungen funktionsfähig ist.

Weiter ist das \textit{Rückverfolgen} der Fahrzeugposition über separate
Encoder überflüssig geworden. Mit der Massnahme, die tendenziell eher
überdimensionierten Motoren bei Positionierungsaufgaben voll zu bestromen, sind
eventuelle Schrittverluste kein Problem mehr. Deshalb ist es zum bestimmen der
zurückgelegten Strecke absolut ausreichend die getätigten Schritte der
Schrittmotoren seit dem letzten Messpunkt auszuwerten. Die Fahrzeugrotation
lässt sich über die Achsgeometrie und die gefahrene Strecke pro Rad ebenfalls
problemlos zurückrechnen. Das Gyroskop ist deshalb auch nicht zum Einsatz
gekommen, obwohl es doch auch mal in Betrieb genommen wurde.

\subsubsection*{Powermanagement}

Der Leistungsbedarf des Roboters lag weit unter den Erwartungen, die aus
Worst-Case-Betrachtungen abgeleitet wurden. Das Konzept an sich hat sich nicht
geändert: Das PowerBoard regelt nach wie vor 12V für das Boardnetz ein,
allerdings hat sich die Akkulaufzeit massiv erhöht.

Die Schaltung auf dem PowerBoard-PCB hat einen Layoutfehler, weshalb der
Schaltungsteil mit der Batterie-Zellenüberwachung nicht in Betrieb genommen
werden konnte. Da der $12V$-Schaltregler auf dem PowerBoard jedoch bereits ab
$12.5V$ Schwierigkeiten hat, die Spannung aufrechtzuerhalten und das System
deshalb abgeschaltet wird, ist die weggefallene Schutzbeschaltung nicht allzu
kritisch. $12.5V$ bei einem 4S-LiPo, wie wir ihn einsetzen, ergibt eine
Zellspannung von $3.125V$, vorausgesetzt, die Zellen sind alle gleichmässig
entladen. Das liegt noch über den Spezifikationen zur Tiefentladung des Akkus.

\subsubsection*{Kommunikation}

Die RS422-Schnittstelle hat sich während der Entwicklung als hinderlich
erwiesen. Ein einfacher UART-zu-USB-Wandler kann mit einem solchen IC nicht
eingesetzt werden, da der UART-Rx-Pin durch den RS422-Receiver permanent auf
High getrieben wird.

Zu Entwicklungszwecken wurde dieser IC deshalb zunächst wieder ausgelötet. Da
sich während der Entwicklung und Erprobung des Systems jedoch keine
Schwierigkeiten oder fehlerhaft übertragenen Datenpakete ergaben, wurde auch im
finalen System auf diese Schnittstelle verzichtet. Mit einer 8-bit-CRC ist eine
Form der Übertragungssicherheit und Fehlererkennung auch im
\textit{prain\_uart-Protokoll} implementiert. Sollte ein Datenpaket falsch
ankommen, wird der Absender darüber informiert und sendet das Paket erneut.
% Verweis auf Prain-UART-Anhang

\subsubsection*{PCBs}

Da beim MotionController einige Sensoren eingespart werden konnten, verfügt
dieser nun über genügend freie Ein- und Ausgänge, um die Firmware des
GripControllers parallel zur MotionController-Firmware auf dem MotionController
auszuführen. Die neue Steuer-Topologie besteht nun nicht mehr aus drei Platinen
(\textit{RaspberryHat, MotionController und GripController}), sondern nur noch
aus den ersten beiden.

Nichtsdestotrotz war der GripController während der Entwicklungsphase eine
grosse Hilfe, da die beiden Teammitglieder aus dem Elektronikbereich so
eigenständiger arbeiten und testen konnten.

\subsubsection*{Akku-Ladestandsanzeige}

Die Ladestandsanzeige auf dem PowerBoard wurde nie in Betrieb genommen. Die
Akkulaufzeit hat sich als völlig unkritisch erwiesen. Eine permanente
Kontroller der Akkuspannung und damit des Ladezustands ist darum überflüssig
geworden währrend der Entwicklungsarbeit.

\subsubsection*{Notstopp - Konzept}

Während im Pren-1-Konzept noch vorgesehen war, die Batterie vom System zu
trennen, wenn der Notstopp betätigt wird, ist dies nun in der Software des
Motion Controllers umgesetzt.

Bei der Inbetriebnahme hat sich gezeigt, dass der Systemzustand, wenn während
der Fahrt der die Batterie getrennt wird, nicht mehr genau vorhergesagt werden
kann. Wenn das Boardnetz wieder eingeschaltet wird, kann es passieren, dass der
Roboter direkt wieder losfährt. Dies und das unsanfte Herunterfahren des
Raspberry Pis sind der Grund dafür, dass der Notstopp nun in Software
implementiert ist. Die Struktur der Firmware des Motion Controllers erlaubt
Broadcasts an alle Tasks. Wird der Notstopp betätigt, werden alle Tasks
angewiesen, alle Bewegungen einzustellen. Danach wird der Motion-Controller
intern über ein Event-Flag daran gehindert, Bewegungen auszuführen, bis der
Notstopp wieder losgelassen wird. Der RaspberryHat wird über das
prain-uart-Protokoll über dieses Ereignis informiert. Dies stellt sicher, dass
sich der gesamte Roboter in einem sicheren Zustand befindet, bis der Notstopp
wieder deaktiviert wird.

\end{document}
