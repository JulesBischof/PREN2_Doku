\documentclass[main.tex]{subfiles} % Subfile-Class

%==============================================================================%
%                                   Subfile                                    %
%==============================================================================%

\begin{document}

% Template

\subsubsection{Änderungen zu Konzpet PREN 1}

\subsubsection*{Sensorik}

\subsubsection*{Powermanagement}

Der Leistungsbedarf des Routers lag weit unter den Erwartungen, die aus
Worst-Case-Betrachtungen abgeleitet wurden. Das Konzept an sich hat sich nicht
geändert: Das PowerBoard regelt nach wie vor 12V für das Boardnetz ein,
allerdings hat sich die Akkulaufzeit massiv erhöht.

Die Schaltung auf dem PowerBoard-PCB hat einen Layoutfehler, weshalb der
Schaltungsteil mit der Batterie-Zellenüberwachung nicht in Betrieb genommen
werden konnte. Da der $12V$-Schaltregler auf dem PowerBoard jedoch bereits ab
$12.5V$ Schwierigkeiten hat, die Spannung aufrechtzuerhalten und das System
deshalb abgeschaltet wird, ist die weggefallene Schutzbeschaltung nicht allzu
kritisch. $12.5V$ bei einem 4S-LiPo, wie wir ihn einsetzen, ergibt eine
Zellspannung von $3.125V$, vorausgesetzt, die Zellen sind alle gleichmäßig
entladen. Das liegt noch über den Spezifikationen zur Tiefentladung des Akkus.

\subsubsection*{Kommunikation}

Die RS422-Schnittstelle hat sich während der Entwicklung als hinderlich
erwiesen. Ein einfacher UART-zu-USB-Wandler kann mit einem solchen IC nicht
eingesetzt werden, da der UART-Rx-Pin durch den RS422-Receiver permanent auf
High getrieben wird.

Zu Entwicklungszwecken wurde dieser IC deshalb zunächst wieder ausgelötet. Da
sich während der Entwicklung und Erprobung des Systems jedoch keine
Schwierigkeiten oder fehlerhaft übertragenen Datenpakete ergaben, wurde auch im
finalen System auf diese Schnittstelle verzichtet. Mit einer 8-bit-CRC ist eine
Form der Übertragungssicherheit und Fehlererkennung auch im
\textit{prain_uart-Protokoll} implementiert. Sollte ein Datenpaket falsch
ankommen, wird der Absender darüber informiert und sendet das Paket erneut.
% Verweis auf Prain-UART-Anhang

\subsubsection*{PCB}

Da beim MotionController einige Sensoren eingespart werden konnten, verfügt
dieser nun über genügend freie Ein- und Ausgänge, um die Firmware des
GripControllers parallel zur MotionController-Firmware auf dem MotionController
auszuführen. Die neue Steuer-Topologie besteht nun nicht mehr aus drei Platinen
(\textit{RaspberryHat, MotionController und GripController}), sondern nur noch
aus den ersten beiden.

Nichtsdestotrotz war der GripController während der Entwicklungsphase eine
große Hilfe, da die beiden Teammitglieder aus dem Elektronikbereich so
eigenständiger arbeiten und testen konnten.

\subsubsection*{Akku-Ladestandsanzeige}

Die Ladestandsanzeige auf dem PowerBoard wurde nie Inbetrieb genommen.

\end{document}
