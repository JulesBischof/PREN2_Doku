\documentclass[main.tex]{subfiles} % Subfile-Class

%==============================================================================%
%                                   Subfile                                    %
%==============================================================================%

\begin{document}

% Template

\subsubsection{Power Management}~\label{sec:PowerManagement}

\subsubsection*{Akku}

Der eingesetzte Akku ist ein 4S-1300mAh LiPo - Akku. Die Kapazität wurde nach
Worst-Case Bedingungen ausgelegt. So wurde überlegt, dass der Raspberry Pi
permanent unter vollast läuft und die Motoren permanent ihren Nennstrom
beziehen. Das ist in der Praxis natürlich nie gegeben. Entsprechend ist der
Akku ein wenig überdimensionert - das Gewichtsbudget lässt diesen Akku
allerdings zu, deshalb ist nicht auf die grosse Kapazität verzichtet.

% ABBILDUNG AKKU

Die Akkulaufzeit beträgt nun mehrere Stunden bei normalem Versuchbetrieb. Die
Anforderungen, dass der Akku für $15min$ Betrieb ausreichen soll, wird also bei
weitem erreicht.

\subsubsection*{Akku - Entladeschutz}

Das PowerBoard beinhaltet eine Batterieschutz. Aufgrund eines Layout-Fehlers
wurde diese allerdings nie in Betrieb genommen. Der Schaltregler für die
12V-Spannung hat bereits bei einer Akkuspannung von ca. 12.5V Probleme damit,
die Boardnetzspannung aufrecht zu halten. Da dies noch oberhalb der kritischen
Entladespannung des Akkus liegt, ist vorerst auf eine entsprechende separate
Schutzbeschaltung verzichtet.

Das Implementieren einer Ladestandskontrolle des Akkus über den RaspberryPi
stellte lediglich eine Wunschanforderung dar. Da die Zeit für andere, höher
Priorisierte Themen benötigt wurde, ist auf diese Funktionalität verzichtet
worden.

\subsubsection*{Boardnetz}

Das Fahrzeug wird mit einem 12V-Boardnetz betrieben. Dazu wird auf dem eigens
dafür erstellten \textit{PowerBoard}-PCB die Spannung aus dem LiPo Akku auf 12V
eingeregelt.

Jeder PCB hat seine eigene Spanungsregelung von 12V auf zunächst 5V5 und
anschliessend noch auf 3V3.

% TODO ABBILDUNG PREN1 SPEISUNG GESAMTSYSTEM

Dadurch ist sichergestellt, dass auch obwohl grosse Verbraucher wie z.B. die
Motoren anlaufen, keine starken Spannungseinbrüche am Raspberry Pi oder an
einer der Microcontroller merkbar sind. Die Spannungsversorgung hat also viel
\textit{Reserve}, um solche sporadischen Schwankungen auszugleichen.

Die Batterielaufzeit wurde mal auf $15 - 20 min$ ausgelegt, wobei vom
Worst-Case ausgegangen wurde. In der Praxis hat sich nun gezeigt, dass dies zu
kritische Betrachtung ist und der Akku nun auch für mehrere Stunden ausreichend
Leistung liefert.

\subsubsection*{Power Management der Motoren}

Die eingesetzten Motorentreiber, die \textit{Tmc5240} von \textit{ADI-Trinamic}
bieten eine Parametrierung des Motorenstroms. So ist es problemlos möglich, im
laufe des Betriebs den Maximalstrom für die jeweiligen Motoren einzustellen.
Für diese Steuerung sind die Ströme nun so ausgelegt, dass sie bei Dreh oder
Positionierungsbewegungen, bei welchen Schrittverluste sehr kritisch sind, den
vollumfänglichen Strom zur verfügung gestellt bekommen, wohingegen sie bei
\textit{Segelfahrten}, also beim Folgen der Linie auf nur in etwa $30\%$ des
Maximalstroms reduziert werden können.

Befindet sich das Fahrzeug dafür im Stillstand, zum Beispiel wenn ein
Knotenpunkt erkannt wurde, oder gerade ein Hindernis aufgenommen wird, so wird
der Motorenstrom sogar ganz abgestellt.

Durch diese Einsparungen liess sich der Leistungsbedarf des Roboters nochmals
enorm reduzieren.

\subsubsection*{das PowerBoard}
Das PowerBoard-PCB hat die Hauptaufgabe, die Batteriespannung von $12.5V bis
    16.8V$ auf eine Boardnetzspannung von $12V$ einzuregeln. Daneben bietet es
einen Labornetzteilanschluss mit automatischer Quellenumschaltung. Wird ein
solches angeschlossen, wird die Speisung über das Netzteil bevorzugt. Neben
dieser Funktion bietet dieses PCB eine gewisse Einschalt-Stromdrosselung, um den
Strom, welcher beim Einschalten des Roboters in die Kondensatoren schiesst, zu
begrenzen.

Dieses PCB hat acht 12V-Steckplätze, an dem alle grösseren Verbraucher des
Systems angeschlossen werden können.

Die integrierte Batteriezellenüberwachung hat einen Layoutfehler und konnte
deshalb nicht inbetriebgenommen werden. Wie bereits vorgängig beschrieben ist
das kein allzu kritisches Problem, da die Sicherheit des Akkus nach wie vor
gewährleitet ist. % Referenz auf Abschnitt

\end{document}