\documentclass[main.tex]{subfiles} % Subfile-Class

%==============================================================================%
%                                   Subfile                                    %
%==============================================================================%

\begin{document}

% Template

\subsubsection{MotionController Firmware}

Die Firmware basiert auf dem Open-Source-Betriebssystem \textit{FreeRTOS}.
Dieses ermöglicht die Ausführung eines einfachen Multithreading-Systems auf
einem Mikrocontroller.

\subsubsection*{Tasks}

Das System ist in verschiedene Tasks unterteilt, die jeweils eine eigene
Funktion erfüllen. FreeRTOS verteilt die Rechenzeit auf die einzelnen Tasks in
einem Time-Slicing Verfahren. Die eingesetzten Tasks sind nachfolgend
aufgeführt:

\begin{description}

    \item[RaspberryHatComTask] Dieser Task übernimmt die gesamte Kommunikation mit dem
        RaspberryHat und ist der einzige Task, der auf die entsprechende
        UART-Schnittstelle zugreift.

    \item[GripControllerComTask] Dieser Task erfüllt praktisch die gleiche Aufgabe wie
        der RaspberryHatComTask, jedoch für den GripController über einen separaten
        UART-Kanal.

    \item[BarrierHandlerTask] In diesem Task ist eine Zustandsmaschine implementiert, die
        den gesamten Bewegungsablauf zur Umplatzierung eines Hindernisses steuert.

    \item[LineFollowerTask] Hier sind alle Funktionen zur Motoransteuerung enthalten.
        Neben dem Linienfolge-Regelalgorithmus verarbeitet dieser Task auch
        \textit{manuelle} Steuerbefehle wie \textit{Stop}, \textit{Move $[xxx]$ mm} und
        \textit{Turn $[Deg]$ °}.

    \item[SensorTask] Dieser Task fragt zyklisch den Ultraschallsensor am vorderen Ende
        des Fahrzeugs ab. Zusätzlich führt er Berechnungen im Rahmen einer
        zeitdiskreten Filterung für den Ultraschallsensor durch. Weitere Details zu
        diesem Filter sind im Anhang~\ref{apdx:FilterDimensionierungHcSr04}
        beschrieben.

    \item[MessageDispatcherTask] Dieser Task analysiert die \texttt{DispatcherMessage},
        die im folgenden Abschnitt näher erläutert wird, und übernimmt die Verteilung
        der Nachrichten. Dadurch erhält jeder Task eine einzige, einheitliche und klar
        definierte Schnittstelle, über die er Kommandos empfangen und senden kann.
        Zudem ermöglicht dieser \textit{Verteil-Task} Broadcast-Nachrichten, wodurch
        beispielsweise alle Tasks gleichzeitig über einen \textit{Stop}-Befehl
        informiert werden.
\end{description}

Die Kommunikation zwischen diesen Tasks erfolgt über Queues, über die Pakete
vom Typ \texttt{DispatcherMessage} ausgetauscht werden. Dabei handelt es sich
um ein Datenpaket, das den Absender, den Empfänger, ein Kommando sowie eine
64-Bit-Payload für Daten enthält. Eine detaillierte Beschreibung findet sich im
Anhang~\ref{apdx:DispatcherMessage}.

\subsubsection*{Steuerungsablauf bei Hindernissen}

Der Bewegungsablauf wird von einer Zustandsmaschine gesteuert, die im
BarrierHandler-Task implementiert ist. Nachfolgend wird lediglich der
Bewegungsablauf beschrieben. Genauere Details zu dieser Zustandsmaschine sind
im Anhang~\ref{apdx:HandlerBarrierStm} dokumentiert.

\begin{enumerate}
    \item Hindernisse, die sich näher als $500$ mm befinden, werden überwacht. Wird ein
          Hindernis in Abbremsreichweite erkannt, wird das Fahrzeug zunächst leicht
          verlangsamt, um die Samplingrate des Ultraschallsensors zu erhöhen.

    \item Sobald sich das Hindernis in einer Distanz befindet, die dem Bremsweg
          entspricht, wird das Fahrzeug abgebremst.

    \item Nachdem das Fahrzeug zum Stillstand gekommen ist, wird aus einer Messreihe von
          Distanzwerten erneut der Median berechnet und die Position ein letztes Mal
          korrigiert.

    \item Sobald sich der Greifer korrekt über dem Hindernis befindet, wird der
          GripController angewiesen, das Hindernis aufzunehmen.

    \item Ist das Hindernis gegriffen, bewegt sich das Fahrzeug eine Fahrzeuglänge
          vorwärts.

    \item Das Fahrzeug wird um $180^\circ$ gedreht.

    \item Der GripController wird angewiesen, das Hindernis abzusetzen.

    \item Nach Erhalt der Bestätigung des GripControllers bewegt sich das Fahrzeug ein
          kleines Stück zurück (Sicherheitsabstand) und dreht sich erneut um $180^\circ$.

    \item Das Hindernishandling ist abgeschlossen, und die Fahrt kann fortgesetzt werden.
          Der LineFollowerTask wird angewiesen, die Linienverfolgung fortzusetzen.
\end{enumerate}

Die Steuerung des Fahrzeugs, einschliesslich Drehungen und Positionsänderungen,
erfolgt ausschliesslich über die bereits erwähnten \texttt{DispatcherMessage}s,
die vom MessageDispatcher an die jeweiligen Tasks weitergeleitet werden.

Um diese \textit{manuellen} Bewegungen auszuführen, ist im LineFollowerTask
eine eigene Zustandsmaschine implementiert, die die Kommandos \texttt{Stop},
\texttt{Move} und \texttt{Turn} ausführen kann. Weitere Details zu diesem
Zustandsautomaten sind im Anhang~\ref{apdx:MovePositionModeStm} beschrieben.

\subsubsection*{Ansteuerung der Schrittmotoren}

Es werden die aus PREN1 evaluierten Schrittmotoren zusammen mit den
Motorentreibern \textit{ADI-Trinamic TMC5240} eingesetzt. Diese Motorentreiber
bieten eine einfache Schnittstelle zur präzisen Steuerung der Motoren. Für die
Linienfolgefunktion werden die Motoren im \textit{Velocity Mode} betrieben, bei
dem der Treiber eine vorgegebene Drehzahl automatisch einregelt.
Positionierungsaufgaben werden im \textit{Position Mode} ausgeführt, bei dem
dem internen Schrittzähler des Motorentreibers Sollpositionen vorgegeben
werden, die er dann selbstständig anfährt.

Ein Motor wird gestoppt, indem ihm im \textit{Velocity Mode} eine Soll-Drehzahl
von $0 \frac{\mu \text{steps}}{t}$ vorgegeben wird.

Die Motorentreiber sind über SPI mit dem MotionController verbunden. Über diese
Schnittstelle werden sämtliche Konfigurationen und Befehle an die
Motorentreiber übertragen.

% ABBILDUNG ADI TMC5240 EVAL BOARD PREN1

\end{document}
