\documentclass[main.tex]{subfiles} % Subfile-Class

%==============================================================================%
%                                   Subfile                                    %
%==============================================================================%

\begin{document}

% Template

\subsubsection{Grip Controller Firmware}

Die Firmware des Grip Controllers basiert wie die des Motion Controllers auf dem Echtzeitbetriebssystem FreeRTOS.\\
Das ursprüngliche Konzept sah vor, den Grip Controller getrennt vom Motion Controller auf einem
eigenen Print zu betreiben. Diese modulare Trennung sollte eine klare funktionale Trennung
sowie eine parallele Entwicklung und Verifikation beider Teilsysteme ermöglichen. Bereits in der Planungsphase
wurden dennoch Anschlüsse auf dem Motion Controller Print vorgesehen, um allenfalls die Funktion des
Grip Controllers zu übernehmen. Um Gewicht einzusparen und um die Systemarchitektur zu vereinfachen,
wurde entschieden, die Firmware des Grip Controllers in die Motion Controller Firmware
zu integrieren. Durch den Wegfall der UART-Kommunikationsschnittstelle zwischen den Controllern
konnte die Komplexität der Signalübertragung deutlich reduziert werden und die Grip Controller Firmware fällt dadurch
deutlich kleiner aus.\\

Die Tasks für die serielle Kommunikation und für den Testbetrieb in der Grip Controller Firmware sind im Endprodukt aufgrund
der zusammenschliessung von Motion Controller und Grip Controller nicht mehr notwendig. Sie bleiben aber 
in der Firmware-Struktur erhalten. Dies hat den Grund, dass allfällige Überarbeitungen der Grip Controller
Firmware unabhängig von dem Motion Controller implementiert und getestet werden können.\\

\subsubsection*{Tasks}
Im folgendem Abschnitt wird nur auf den Task der Firmware eingegangen, welcher im Endprodukt enthalten ist.
Die vollständige Firmware mit allen Tasks ist im Anhang~\ref{apdx:FirmwareGripController} genauer beschrieben.

\begin{description}

    \item[ServoDriveTask] Dieser Task übernimmt das komplette Hindernishandling. Er wartet auf die Befehle "CraneGrip"
    und "CraneRelease" vom Motion Controller und sendet seinerseits ein "GcAck" zurück, falls das Hindernis gegriffen oder
    abgestellt wurde.
\end{description}

Die Kommunikation zwischen dem Motion Controller und dem Grip Controller erfolgt über Queues von FreeRTOS. Im Anhang~\ref{XYZ} wird genauer auf den
Informationsaustausch der beiden Firmwares eingegangen.

\end{document}
