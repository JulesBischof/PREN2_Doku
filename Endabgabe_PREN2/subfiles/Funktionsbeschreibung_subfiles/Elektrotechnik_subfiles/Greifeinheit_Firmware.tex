\documentclass[main.tex]{subfiles} % Subfile-Class

%==============================================================================%
%                                   Subfile                                    %
%==============================================================================%

\begin{document}

% Template

\subsubsection{Grip Controller Firmware}

Die Firmware des GripControllers basiert wie die des MotionControllers auf dem Echtzeitbetriebssystem FreeRTOS.\\
Das ursprüngliche Konzept sah vor, den GripController getrennt vom MotionController auf einem
eigenen Print zu betreiben. Diese modulare Trennung sollte eine klare funktionale Trennung
sowie eine parallele Entwicklung und Verifikation beider Teilsysteme ermöglichen. Bereits in der Planungsphase
wurden dennoch Anschlüsse auf dem MotionController Print vorgesehen, um allenfalls die Funktion des
GripControllers zu übernehmen. Um Gewicht einzusparen und um die Systemarchitektur zu vereinfachen,
wurde entschieden, die Firmware des GripControllers als externe Datei in die MotionController Firmware
zu integrieren. Durch den Wegfall der UART-Kommunikationsschnittstelle zwischen den Controllern
konnte die Komplexität der Signalübertragung reduziert werden und die GripController Firmware fällt dadurch
deutlich kleiner aus.\\

Die Tasks für die serielle Kommunikation und für den Testbetrieb in der Firmware sind im Endprodukt aufgrund
der Zusammenschliessung nicht mehr notwendig. Sie bleiben aber 
in der Firmware-Struktur erhalten. Dies hat den Grund, dass allfällige Überarbeitungen der GripController
Firmware unabhängig implementiert und getestet werden können.\\

\subsubsection*{Tasks}
Im folgendem Abschnitt wird nur auf den Task der Firmware eingegangen, welcher im Endprodukt enthalten ist.
Die vollständige Firmware mit allen Tasks ist im Anhang~\ref{apdx:FirmwareGripController} genauer beschrieben.

\begin{description}
    \item[ServoDriveTask] Dieser Task ist für die Steuerung des Greifmechanismus verantwortlich. Er wartet auf
    Kommandos vom MotionController und sendet seinerseits ein Kommanda wenn ein Hindernis gegriffen oder abgesetzt
    wurde.
\end{description}


\subsubsection{Steuerung des Greifers}
Die Steuerung des Greifvorgangs erfolgt über eine interne State Machine, welche die verschiedenen Phasen des
Ablaufs in klar definierte Zustände unterteilt. Die Zustandsübergänge werden durch eingehende Befehle gesteuert.
Durch interne Statusflags ist dem Greifer immer bekannt in welcher Endlage er sich befindet und ob er ein Hindernis
gegriffen hat oder nicht. Dadurch ist eine sequentielle Abarbeitung der Bewegungssequenzen gewährleistet. Eine
ausführliche Beschreibung der Zustandsmaschine befindet sich im Anhang~\ref{apdx:FirmwareGripController}



\subsubsection{Kommunikation mit MotionController}
Die Kommunikation zwischen MotionController und GripController erfolgt über FreeRTOS-Queues. Der ServoDriveTask empfängt
dabei die Kommandos „CraneGrip“ und „CraneRelease“ über die interne Schnittstelle und sendet nach erfolgreichem Ablauf ein
„GcAck“ als Rückmeldung an die MotionController-Firmware. Eine gesonderte UART-Verbindung ist durch die Integration der
beiden Firmwares nicht mehr erforderlich. Im Anhang~\ref{apdx:ComGripControllerMotionController} wird genauer auf den Informationsaustausch der beiden Firmwares
eingegangen.

\end{document}
