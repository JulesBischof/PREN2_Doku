\documentclass[main.tex]{subfiles} % Subfile-Class

%==============================================================================%
%                                   Subfile                                    %
%==============================================================================%

\begin{document}

% Template

\subsubsection{Sensorik}

\subsubsection*{Liniensensor}

Der Liniensensor aus dem Konzept aus PREN1 hat sich als sehr zuverlässig
erwiesen. Er arbeitet mit UV-Licht, das das weisse Klebeband auf den Fugen
fluoreszieren lässt, während die Fugen zwischen den Fliesen nicht
fluoreszieren. Die Beleuchtungsstärke wird im sichtbaren Spektrum ausgewertet,
sodass ausschliesslich das fluoreszierende Klebeband erfasst wird. Dies
ermöglicht einen starken Kontrast zur Umgebung.

% TODO Abbildung aus Pren 1 Schema

Seine Dimensionierung basiert auf der Sättigung eines Phototransistors. Solange
sich der Sensor über einer Fliese befindet, bleibt der Phototransistor wenig
leitend. Dadurch wird der Spannungspegel hauptsächlich durch den Widerstand auf
HIGH gezogen. Der Widerstand wurde so gewählt, dass der Transistor in Sättigung
gerät, sobald sich ein Klebeband unterhalb des Sensors befindet. In diesem Fall
wird der Spannungspegel stark gegen GND gezogen, sodass eine Linie eindeutig
detektiert werden kann.

Das Gehäuse dieses Sensors schirmt jede Messzelle einzeln stark von der
Umgebung ab. Dadurch beeinflussen sich die Messzellen weder gegenseitig, noch
werden sie durch Sonnenlicht oder andere Umwelteinflüsse gestört.

Aus Sicherheitsgründen ist die Elektronik, die diesen Sensor ansteuert, in der
Lage, die UV-LEDs ein- und auszuschalten. Diese werden nur aktiviert, wenn eine
Messung erfolgt. Falls nach zehn Messungen keine Linie gefunden wird, wird der
Messprozess abgebrochen und die UV-LEDs bleiben deaktiviert. Befindet sich der
Roboter nicht in Bewegung, werden also maximal zehn UV-Impulse ausgesendet, was
einer Zeitdauer von $\approx 100$ ms entspricht.

Die Sensordaten werden über einen Analog-Digital-Wandler ausgewertet. Nachdem
alle Zellen ausgelesen wurden, werden die Messwerte zunächst auf einen Bereich
von $0 \dots 1000$ normiert, um Bauteiltoleranzen zwischen den Zellen zu
kompensieren. Anschliessend werden die Werte \textit{gewichtet aufsummiert}.
Wie die Ermittlung der Linienposition funktioniert und wie dieser Wert in der
Regelung eingesetzt wird, ist im Anhang~\ref{apdx:LineFollowerRegler} im Detail
beschrieben.

% BILD Liniensensor

% Verweis Anhang Liniensensor

\subsubsection*{Streckenrückverfolgung – Distanz und Rotation}

Die gefahrene Strecke kann vollständig über die registrierten Schritte der
Schrittmotoren bestimmt werden. Dazu wurde die Motorbeschleunigung auf ein Mass
reduziert, das einen Schlupf der Räder verhindert. Wie die zurückgelegte
Distanz der Räder von der Firmware erfasst wird, ist im
Anhang~\ref{apdx:Distanz_Tracking} detailliert beschrieben. Die erfasste
Strecke erlaubt zudem Rückschlüsse auf die durchgeführte Rotation des
Fahrzeugs.

\subsubsection*{Erkennung von Hindernissen}

Im ursprünglichen Konzept aus PREN1 wurde die Hinderniserkennung redundant
ausgeführt: Ein Ultraschallsensor sollte die Annäherung an ein Hindernis
erkennen, während eine Laserlichtschranke den endgültigen
\textit{Greifzeitpunkt} bestimmen sollte. Aus Kostengründen wurde zunächst eine
Lösung realisiert, die ausschliesslich auf dem Ultraschallsensor basiert. Mit
verschiedenen statistischen und systemtheoretischen Methoden, die im
Anhang~\ref{apdx:FilterDimensionierungHcSr04} näher erläutert werden, lässt
sich die Position eines Hindernisses allein mit dem Ultraschallsensor sehr
genau vorhersagen.

Ein wesentliches Problem des eingesetzten Sensors ist die niedrige Abfragerate
für Distanzwerte. Der implementierte Filter ermöglicht jedoch eine Vorhersage
der Hindernisposition basierend auf der aktuellen Fahrzeuggeschwindigkeit.
Dadurch können auch rechnerisch ermittelte Hindernispositionen zwischen zwei
Messpunkten bestimmt werden. Nach jeder Messung wird das Zustandsmodell
korrigiert, abhängig von der Genauigkeit der vorherigen Schätzung. Dieses
Filter wird im Anhang~\ref{apdx:Adaptiver_Tiefpass} hergeleitet und
parametriert.

% TODO\ Bild Fahrzeug vor Hindernis

\subsubsection*{Pylonen LIDAR}

GABRIEL/SANDRO .. ?

\subsubsection*{Kamera}

GABRIEL

\subsubsection*{Greiferposition}

MANUEL

\end{document}
