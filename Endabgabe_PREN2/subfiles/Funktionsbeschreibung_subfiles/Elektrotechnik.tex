\documentclass[main.tex]{subfiles} % Subfile-Class

%==============================================================================%
%                                   Subfile                                    %
%==============================================================================%

\begin{document}

% Template

\subsection{Elektrotechnik}

Die elektrische Architektur des Fahrzeugs ist modular aufgebaut und trennt bewusst 
zwischen leistungsführender Hardwareansteuerung und übergeordneter Navigationslogik. 
Ziel war es, eine robuste, fehlertolerante und servicefreundliche Elektronikstruktur 
zu realisieren, die gleichzeitig den Anforderungen an Echtzeitfähigkeit, 
Energieeffizienz und Gewicht entspricht.

Kern des Systems ist ein aus drei Funktionsgruppen bestehendes Steuerungskonzept: 
ein Raspberry Pi zur Zielwahl und Bildverarbeitung, ein MotionController zur 
Ansteuerung aller Aktoren sowie ein PowerBoard zur Spannungsaufbereitung. Die 
Kommunikation zwischen den Einheiten erfolgt über ein eigens entwickeltes, 
minimalistisches UART-Protokoll mit integrierter Fehlerprüfung. Sensoren und Motoren 
sind gezielt aufgeteilt, um kurze Signalwege und eine klare Zuordnung zu gewährleisten.

Die folgenden Abschnitte beschreiben den Aufbau und die Aufgabenverteilung der 
elektrischen Komponenten, ihre Verschaltung sowie die softwareseitige Ansteuerung 
der verschiedenen Teilsysteme.

\subfile{./Elektrotechnik_subfiles/Elektronik_Gesamtuebersicht.tex}
\newpage

\subfile{./Elektrotechnik_subfiles/Sensorik.tex}
\newpage

\subfile{./Elektrotechnik_subfiles/Antriebe.tex}
\newpage

\subfile{./Elektrotechnik_subfiles/Antriebsregelung_Firmware.tex}
\newpage

\subfile{./Elektrotechnik_subfiles/Greifeinheit_Firmware.tex}
\newpage

\subfile{./Elektrotechnik_subfiles/Power_Management.tex}
\newpage

\subfile{./Elektrotechnik_subfiles/Aenderungen_zu_konzept_PREN1.tex}
\newpage
\end{document}
