\documentclass[main.tex]{subfiles} % Subfile-Class

%==============================================================================%
%                                   Subfile                                    %
%==============================================================================%

\begin{document}

% Template

\subsection{Elektrotechnik}

Die elektrische Architektur des Fahrzeugs ist modular aufgebaut und trennt bewusst 
zwischen leistungsführender Hardwareansteuerung und übergeordneter Navigationslogik. 
Ziel war es, eine robuste und servicefreundliche Elektronikstruktur 
zu realisieren, welche eine parallele Entwicklung und Verifikation erlaubte.
Ausserdem galt die Anforderung, dass das System Echtzeitfähig ist und nicht
zu viel Gewicht beansprucht.

Kern des Systems ist ein aus drei Funktionsgruppen bestehendes Steuerungskonzept: 
ein Raspberry-HAT zur Zielwahl, Navigation und Bildverarbeitung, ein MotionController zur 
Ansteuerung aller Aktoren sowie ein PowerBoard zur Spannungsaufbereitung. Die 
Kommunikation zwischen den Einheiten erfolgt über ein eigens entwickeltes 
UART-Protokoll.

Die folgenden Abschnitte beschreiben den Aufbau und die Aufgabenverteilung der 
elektrischen Komponenten, ihre Verschaltung sowie die softwareseitige Ansteuerung 
der verschiedenen Teilsysteme.

\subfile{./Elektrotechnik_subfiles/Elektronik_Gesamtuebersicht.tex}
\newpage

\subfile{./Elektrotechnik_subfiles/Sensorik.tex}
\newpage

\subfile{./Elektrotechnik_subfiles/Antriebe.tex}
\newpage

\subfile{./Elektrotechnik_subfiles/Antriebsregelung_Firmware.tex}
\newpage

\subfile{./Elektrotechnik_subfiles/Greifeinheit_Firmware.tex}
\newpage

\subfile{./Elektrotechnik_subfiles/Power_Management.tex}
\newpage

\subfile{./Elektrotechnik_subfiles/Aenderungen_zu_konzept_PREN1.tex}
\newpage
\end{document}
