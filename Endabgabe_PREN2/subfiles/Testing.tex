\documentclass[main.tex]{subfiles} % Subfile-Class

%==============================================================================%
%                                   Subfile                                    %
%==============================================================================%

\begin{document}

% Template

\section{Testing}
In diesem Kapitel werden die wichtigsten Teilsysteme des Fahrzeugs
getestet. Da die Entwicklung laufend angepasst und verbessert wurde,
macht eine testdatum-basierte Darstellung wenig Sinn. Stattdessen
wurde jeweils geprüft, ob die definierten Sprintziele erreicht wurden.
Dadurch konnte direkt im Entwicklungsprozess verifiziert werden, ob
ein System wie geplant funktioniert oder noch Optimierungen im nächsten
Sprint nötig sind.

\subsection*{Linienverfolgung}
Die Funktionsfähigkeit des Liniensensors in Kombination mit dem
Antriebssystem wurde iterativ über mehrere Sprints hinweg überprüft.
In Sprint 1 lag der Fokus auf dem zuverlässigen Erkennen und Folgen
der Linie sowie dem Erkennen von Knotenpunkten. Zu diesem Zweck
wurden erste Fahrtests durchgeführt, um zu beobachten, wie das
Fahrzeug auf die Reglerausgabe reagiert. Dabei zeigte sich, dass der
Regler zu leichten Schwingungen neigte und weiter optimiert werden
musste. Das wiederholte Anhalten auf Knotenpunkten funktionierte
hingegen zuverlässig. Die Tests erfolgten nicht auf dem
Wettkampfuntergrund und dienten primär der funktionalen Umsetzung, nicht
der finalen Abstimmung auf Wettkampfbedingungen.\\
In Sprint 2 wurden verschiedene Optimierungen am Regelsystem
vorgenommen. Ziel war ein schnelleres und stabileres Einregeln des Fahrzeugs
gemäss den Anforderungen des Wettkampfs. Der Regelprozess wurde
aufgezeichnet und anschliessend in MATLAB
analysiert~\ref{apdx:Regler_Parametrierung}.
Die daraus gewonnenen Erkenntnisse ermöglichten eine gezielte
Anpassung der Reglerparameter. Im Anschluss daran erfolgte die Validierung auf
dem vorgesehenen Wettkampfuntergrund, welche erfolgreich verlief.
Während der Tests zeigte sich jedoch, dass einzelne Fugen irrtümlich
als Knotenpunkte erkannt wurden. Im
Anhang~\ref{apdx:Liniensensor_Liniendetektion}
ist die Streuung der Messwerte dargestellt, welche die Unterscheidung
zwischen Fliesen, Fugen und Linie durch den Liniensensor veranschaulicht.
Um eine mögliche Überlappung der Messbereiche von Fuge und Linie zu
reduzieren, wurde der Sensorbereich mit einem lichtdichten Schwamm abgeschirmt.
Dadurch konnte der Kontrast zur tatsächlichen Linie verstärkt und die
fehlerhafte Interpretation der Fugen verhindert werden. In den anschliessenden
Testläufen traten keine Störungen mehr auf.

\subsection*{Hindernishandling}
Die Hindernisbewältigung wurde in Sprint 2 als Ziel definiert. Die
Teilfunktionen wurden nacheinander getestet und schrittweise zum
Gesamtsystem zusammengeführt.\\
Zuerst wurde die Hinderniserkennung mittels Ultraschallsensor
überprüft. In den ersten Tests wurde das Hindernis regelmässig zu spät erkannt
und überfahren. Um dieses Problem zu beheben, musste die
Geschwindigkeit angepasst und ein Zwischenhalt kurz vor dem Hindernis
eingebaut werden.
Details dazu sind im Anhang~\ref{apdx:Sensorverhalten} ersichtlich.
Durch anschliessende Tests konnte die Positionierungsgenauigkeit beim Anfahren
des Hindernisses verbessert und in die gewünschte Toleranz der
Greiferbacken gebracht werden.\\
Das Greifen mit den zwei Servos wurde mehrfach getestet. Der Ablauf
funktionierte stets zuverlässig, das Hindernis wurde nie fallengelassen.
Lediglich die Geschwindigkeit der Servos wurde mehrmals feinjustiert,
um den Greifvorgang möglichst schnell auszuführen.\\
Das Drehen mit aufgenommenem Hindernis stellte aufgrund von Schlupf
zunächst eine Herausforderung dar. Daher wurde beim Testen die
Beschleunigung schrittweise
reduziert, bis eine Drehung mit vertretbarem Schlupf möglich war.\\
Das Abstellen des Hindernisses innerhalb der Toleranz wurde in
mehreren Durchläufen geprüft. Dazu wurde vor der Hindernisbewältigung
ein Markierungsstrich
auf der Linie angebracht. Nach der Bewältigung des Hindernisses wurde
der Abstand zur Markierung gemessen. Auch hier musste die
Beschleunigung schrittweise
zurückgenommen werden, bis der Schlupf keinen gravierenden Einfluss
mehr hatte. Bei den abschliessenden Tests wurde das Hindernis
konstant innerhalb der
vorgegebenen Toleranz abgesetzt.

\subsection*{Notstopp}
Der Notstopp wurde in Sprint 3 implementiert. Nach jeder Erweiterung
der Firmware wurde die Funktion erneut getestet, um sicherzustellen,
dass sie weiterhin korrekt funktioniert.

\subsection*{LIDAR}
Die LIDAR-Integration wurde als eigenständiges Teilziel
festgelegt. Zunächst entstand in Python eine dedizierte
\texttt{LidarSensor}-Klasse, welche die Rohdaten des TF-Luna über
I\textsuperscript{2}C zyklisch ausliest, zwischenspeichert und über eine
einfache API an den High-Level-Controller weitergibt. Parallel dazu
wurde ein Unterprogramm (\textit{ldtest}) erstellt, das den Sensor
losgelöst vom übrigen System betreibt und die gemessenen Distanzen
(plus Flux und Temperatur) in
einem Terminal-Log ausgibt. Mit diesem Werkzeug führten wir eine
systematische Messreihe durch: verschiedene Referenzobjekte (Pylon,
Kartonschachtel, Handfläche) wurden in definierten Abständen aufgestellt
und die zurückgemeldeten Werte mit einem Messband
verglichen. Erst als die Abweichungen durch die Intervalle der
Abfragen im gesamten Arbeitsbereich stabil
innerhalb der erwarteten Toleranz lagen, wurde der Sensor dauerhaft auf
dem Fahrzeug installiert.

In den darauf folgenden Fahrtests zeigte sich jedoch, dass die zuvor
gewählte Montagehöhe zu hoch war: Der Messstrahl traf bei einem Pylon
sehr weit oben auf, was die Detektierung auf Distanz sehr sensibel machte.
Zur Lösung dieses Problems wurde der Sensor soweit wie möglich tiefer
in das Chassis versetzt und zusätzlich ein leichtes
"Sweepen" implementiert. Dabei schwenkt das Fahrzeug bei der
Detektierung um wenige Grad nach links und rechts,
sodass ein breiterer Horizontalbereich abgedeckt wird. Durch diese
Kombination aus angepasster Höhe und Bewegungsmuster konnte die
Erkennungsrate von Pylonen erhöht und die Zuverlässigkeit
der Hindernisdetektion im Parcours bedeutend verbessert werden.

\end{document}
