\documentclass[main.tex]{subfiles} % Subfile-Class

% ============================================================================== %
%                            Subfile document                                    %
% ============================================================================== %

\begin{document}

% Template

\subsection{Risikoanalyse}

Anhang ~\ref{apx:risikoanalyse} zeigt die zu Semesterbeginn erfassten Risiken
auf. Sie wurden nach der ALARP-Methode erfasst und gemeinsam bewertet. Dadurch
konnten kritische Knackpunkte der Aufgabenstellung sehr frühzeitig erfasst
werden und der Fokus konnte während des Semesters auf entsprechende Teilgebiete
gelegt werden.

Durch diese Analyse ist zum Semesterstart der Fokus auf eine saubere Erkennung
der Knotenpunktabgänge über die Kamera gefallen. Die Inbetriebnahme der PCB's
hatte ebenfalls zu Beginn des Semesters höchste Priorität für den
Elektronikbereich, damit das gesamte Projektteam möglichst frühzeitig einen
lauffähigen Roboter zur Implementierung der Teilfunktionen erhält. Auf der
mechanischen Seite wurde viel Zeit in die stabile Auslegung eines leichten
Chassis investiert, wodurch das anfangs kritisch gesehene Zielgewicht von 2 kg
erreicht werden konnte.

\end{document}
