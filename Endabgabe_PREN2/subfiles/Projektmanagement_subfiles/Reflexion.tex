\documentclass[main.tex]{subfiles} % Subfile-Class

%==============================================================================%
%                                   Subfile                                    %
%==============================================================================%

\begin{document}

% Template

\subsection{Reflexion}

Im Rückblick auf das Projekt können wir festhalten, dass sich unser bereits in PREN1
entwickeltes Konzept als durchdacht und tragfähig erwiesen hat. Wir konnten es mit nur
wenigen Anpassungen in die Praxis umsetzen. Die klare modulare Trennung sowie die frühzeitig definierten Schnittstellen
zwischen den Disziplinen haben sich dabei als besonders vorteilhaft erwiesen.\\

Unsere Planung war von Anfang an stark durch die Erfahrungen aus PREN1 geprägt.
Insbesondere das Bewusstsein für interdisziplinäre Abhängigkeiten und die Wichtigkeit
strukturierter Planung hat uns geholfen, das Projekt effizient aufzusetzen. Bereits zu
Beginn des Semesters haben wir zentrale technische und organisatorische Weichen gestellt,
was uns während der Umsetzung ermöglichte, parallel und weitgehend reibungslos zu
arbeiten. Dadurch kam es nur vereinzelt zu Verzögerungen, etwa durch externe
Abhängigkeiten bei der Fertigung oder bei Lieferungen. Die Sprintstruktur hat uns 
geholfen, den Fortschritt regelmässig zu reflektieren und bei Bedarf flexibel auf 
neue Erkenntnisse zu reagieren. Durch den wöchentlichen Austausch im
Team konnten Blockaden früh erkannt und gelöst werden. \\

Gleichzeitig zeigte sich im Gegenschluss, dass der physische Roboter während der aktiven
Entwicklungsphase teilweise als Engpass fungierte. Sowohl Informatik als auch
Elektrotechnik waren auf den laufenden Roboter angewiesen, um ihre jeweiligen Module zu
testen und weiterzuentwickeln. Dies führte in einzelnen Phasen zu Wartezeiten und leicht
ineffizienter Ressourcennutzung. In einem zukünftigen Projekt – oder mit mehr Zeit und
technischer Unterstützung – wäre der Aufbau einer vollständig virtualisierten
Simulationsumgebung von grossem Vorteil. Eine solche Lösung würde nicht nur
ortsunabhängiges Arbeiten ermöglichen, sondern auch parallele Tests und schnellere
Fehleranalysen erlauben.\\

Insgesamt sind wir mit dem Ergebnis unserer Arbeit sehr zufrieden. Die Kombination aus
fundierter Vorarbeit, solider Planung und effektiver Kommunikation hat es uns ermöglicht,
unser Zielprodukt erfolgreich umzusetzen. Darüber hinaus konnten wir als Team wertvolle 
Erfahrungen im interdisziplinären Arbeiten sammeln, die über das Projekt hinaus Wirkung 
zeigen werden.

\end{document}
