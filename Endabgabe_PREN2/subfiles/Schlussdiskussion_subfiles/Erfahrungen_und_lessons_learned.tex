\documentclass[main.tex]{subfiles} % Subfile-Class

%==============================================================================%
%                                   Subfile                                    %
%==============================================================================%

\begin{document}

% Template

\subsection{Erfahrungen und Lessons Learned}

\subsubsection{Mechanik}

\subsubsection{Elektronik}
Im Rahmen der Entwicklung ergaben sich im Bereich der Hardware- und Firmware-Umsetzung verschiedene relevante
Erkenntnisse. So wurde für das Projekt FreeRTOS in Verbindung mit der Programmiersprache C++ eingesetzt. Durch
die praktische Anwendung konnten die Kenntnisse im Umgang mit dem Echtzeitbetriebssystem sowie der objektorientierten
Programmierung gefestigt und erweitert werden. Auch das Bestücken der Leiterplatten, das Prüfen von Signalen
und Spannungen sowie die Fehlersuche an den gefertigten PCBs erwiesen sich als besonders lehrreich.
Neben diesen eher allgemeinen Erkenntnissen traten im Projektverlauf jedoch auch spezifische Herausforderungen auf,
aus denen besonders prägende Lessons Learned hervorgingen, die nachfolgend dargestellt werden:

Obwohl QFN-Gehäuse aufgrund ihrer kompakten Bauweise für platzsparende Designs
vorteilhaft erscheinen, hat sich im praktischen Einsatz gezeigt, dass sie für die manuelle Bestückung
und Nachbearbeitung nur bedingt geeignet sind. Bereits wenn nur ein wenig zu viel Lötzinn auf der
Lötstelle aufliegt, kann es zu Kurzschlüssen führen. Weil die Lötstellen sich auf der Unterseite des
Bauteils befinden, ist keine optische Prüfung auf Kurzschlüsse möglich. Aus diesem Grund sollte in
zukünftigen Projekten auf SOIC-Gehäuse zurückgegriffen werden. Diese verfügen über
seitlich herausgeführte Anschlussbeine, die das manuelle Löten sowie
notwendige Nacharbeiten erheblich erleichtern.

Die Stromaufnahme der Servomotoren wurde in der Entwurfsphase unterschätzt.
Während der Entwicklungsphase zeigten die Servomotoren ein ungewöhnliches Verhalten, das zunächst auf eine
fehlerhafte PWM-Ansteuerung hindeutete. Erst deutlich später stellte sich heraus, dass die Stromversorgung von
maximal 3 A Ausgangsstrom unterdimensioniert war und somit die eigentliche Ursache des Problems dartstellte.
Zukünftig soll die Stromversorgung bei ähnlichen Fehlersymptomen frühzeitig als mögliche Fehlerquelle in Betracht
gezogen werden.



\subsubsection{Informatik}

Text here

\end{document}
