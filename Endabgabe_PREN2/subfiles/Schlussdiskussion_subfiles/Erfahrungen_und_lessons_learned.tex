\documentclass[main.tex]{subfiles} % Subfile-Class

%==============================================================================%
%                                   Subfile                                    %
%==============================================================================%

\begin{document}

% Template

\subsection{Erfahrungen und Lessons Learned}

\subsubsection{Mechanik}
Die ursprünglich in PREN 1 vorgesehene Etagenlösung wurde verworfen. Stattdessen wurde eine flache und 
wartungsfreundliche Konstruktion mit nur einer Grundplatte umgesetzt. Durch die Anbringung der 
Motoren und des Powerboards unterhalb der Grundplatte konnte wertvoller Raum auf der Oberseite eingespart 
und die Zugänglichkeit der Komponenten verbessert werden.

Im Bereich der Antriebseinheit zeigte sich, dass es sich lohnt, bestehende Bauteile kritisch zu hinterfragen. 
So wurden zahlreiche Iterationen eines Adapterteils durchgeführt, um die ursprünglich vorgesehenen Felgen 
mit den Motorwellen zu verbinden. Letztlich erwies sich diese Lösung als zu aufwändig und unzuverlässig. 
Stattdessen wurde eine komplett neue Felge mitsamt Motoraufnahme im CAD konstruiert, wodurch lediglich 
der Reifen des Komplettrads weiterverwendet wurde. Diese Entscheidung führte zu einer robusteren und 
funktionaleren Umsetzung und verdeutlichte, dass bestehende Komponenten nicht zwangsläufig übernommen werden 
müssen, wenn eine bessere Eigenkonstruktion möglich ist.

Das Chassis wurde vollständig aus PLA im 3D-Druckverfahren gefertigt. Die Möglichkeit, mehrere Iterationen 
rasch umzusetzen, erwies sich als grosser Vorteil im Entwicklungsprozess. Dabei konnte gezielt Gewicht 
eingespart werden, beispielsweise durch Materialaussparungen und eine kompakte, funktionale Gestaltung. 
Das Fahrzeug blieb trotz vollständiger Integration aller Komponenten deutlich unter dem Gewichtslimit von 2 kg.

Die Montage der Sensorik und der Kamera erforderte eine präzise Abstimmung mit der Informatik. Eine frühere 
Festlegung dieser Schnittstellen hätte Doppelarbeiten und Nachjustierungen reduziert. Insgesamt zeigte sich, 
dass mechanische Änderungen in vielen Fällen Auswirkungen auf die angrenzenden Disziplinen haben, was die 
Bedeutung interdisziplinärer Planung nochmals unterstrich.

\subsubsection{Elektronik}
Im Rahmen der Entwicklung ergaben sich im Bereich der Hardware- und Firmware-Umsetzung verschiedene relevante
Erkenntnisse. So wurde für das Projekt FreeRTOS in Verbindung mit der Programmiersprache C++ eingesetzt. Durch
die praktische Anwendung konnten die Kenntnisse im Umgang mit dem Echtzeitbetriebssystem sowie der objektorientierten
Programmierung gefestigt und erweitert werden. Auch das Bestücken der PCB's, das Prüfen von Signalen
und Spannungen sowie die Fehlersuche an den gefertigten PCBs erwiesen sich als besonders lehrreich.
Neben diesen eher allgemeinen Erkenntnissen traten im Projektverlauf jedoch auch spezifische Herausforderungen auf,
aus denen besonders prägende Lessons Learned hervorgingen, die nachfolgend dargestellt werden:

Obwohl QFN-Gehäuse aufgrund ihrer kompakten Bauweise für platzsparende Designs
vorteilhaft erscheinen, hat sich im praktischen Einsatz gezeigt, dass sie für die manuelle Bestückung
und Nachbearbeitung nur bedingt geeignet sind. Bereits wenn nur ein wenig zu viel Lötzinn auf der
Lötstelle aufliegt, kann es zu Kurzschlüssen führen. Weil die Lötstellen sich auf der Unterseite des
Bauteils befinden, ist keine optische Prüfung auf Kurzschlüsse möglich. Aus diesem Grund sollte in
zukünftigen Projekten auf SOIC-Gehäuse zurückgegriffen werden. Diese verfügen über
seitlich herausgeführte Anschlussbeine, die das manuelle Löten sowie
notwendige Nacharbeiten erheblich erleichtern.

Die Stromaufnahme der Servomotoren wurde in der Entwurfsphase unterschätzt.
Während der Entwicklungsphase zeigten die Servomotoren ein ungewöhnliches Verhalten, das zunächst auf eine
fehlerhafte PWM-Ansteuerung hindeutete. Erst deutlich später stellte sich heraus, dass die Stromversorgung von
maximal 3 A Ausgangsstrom unterdimensioniert war und somit die eigentliche Ursache des Problems dartstellte.
Zukünftig soll die Stromversorgung bei ähnlichen Fehlersymptomen frühzeitig als mögliche Fehlerquelle in Betracht
gezogen werden.

Beim Testen kam es auf einem PCB infolge einer Stromrückspeisung zur Zerstörung des LDO-Reglers, da dieser gegenüber
rückfließenden Strömen sehr empfindlich ist. Die Ursache lag in der getrennten Versorgung der Servomotoren. 
Ohne geeignete Schutzmaßnahmen kam es dadurch zu einem unkontrollierten Rückstrom von den Servomotoren über die
PWM-Pins zum LDO. Wenn mehrere Spannungsversorgungen ein einzelnes System versorgen und es keine andere
Lösungsmöglichkeit gibt, muss unbedingt ein Schutz vor Rückspeisung eingeplant werden, um Bauteilschäden zu vermeiden.



\subsubsection{Informatik}

Text here

\end{document}
