\documentclass[main.tex]{subfiles} % Subfile-Class

%==============================================================================%
%                                   Subfile                                    %
%==============================================================================%

\begin{document}

% Template

\subsection{Erfahrungen und Lessons Learned}

\subsubsection{Mechanik}

\subsubsection{Elektronik}
%Im Rahmen der Entwicklung ergaben sich insbesondere im Bereich der Hardwareumsetzung mehrere
%relevante Erkenntnisse, die für zukünftige Projekte berücksichtigt werden sollen.\\
%Obwohl QFN-Gehäuse aufgrund ihrer kompakten Bauweise für platzsparende Designs grundsätzlich
%vorteilhaft erscheinen, hat sich im praktischen Einsatz gezeigt, dass sie für die manuelle Bestückung
%und Nachbearbeitung nur bedingt geeignet sind. Bereits wenn nur ein wenig zu viel Lötzinn auf der
%Lötstelle ist, kann es zu Kurzschlüssen führen. Die Lötstellen befinden sich auf der Unterseite des
%Bauteils, wodurch keine optische Prüfung auf Kurzschlüsse möglich wird. Aus diesem Grund wird in
%zukünftigen Entwicklungsprojekten verstärkt auf SOIC-Gehäuse zurückgegriffen. Diese verfügen über seitlich herausgeführte Anschlussbeine, die eine optische Inspektion, das manuelle Löten sowie notwendige Nacharbeiten erheblich erleichtern. Dadurch wird eine höhere Prozesssicherheit und Zuverlässigkeit während der Prototypenfertigung erzielt.

%Zudem wurde die Stromaufnahme der verwendeten Servomotoren in der Entwurfsphase deutlich unterschätzt. Die ursprünglich vorgesehene Spannungsversorgung mit einer maximalen Stromabgabe von 3 A erwies sich unter realen Betriebsbedingungen als unzureichend, insbesondere bei gleichzeitiger Ansteuerung mehrerer Servos. Zur Sicherstellung eines stabilen Systembetriebs wurde eine zusätzliche Spannungswandlung implementiert, die eine deutlich höhere Strombelastbarkeit aufweist und so eine zuverlässige Versorgung der Aktoren gewährleistet.

\subsubsection{Informatik}

Text here

\end{document}
