\documentclass[main.tex]{subfiles} % Subfile-Class

%==============================================================================%
%                                   Subfile                                    %
%==============================================================================%

\begin{document}

% Template

\subsection{Fazit und Ausblick}

\subsubsection*{Fazit}
Das Projektmodul PREN~2 diente der praktischen Umsetzung und Validierung des in 
PREN 1 entwickelten Konzepts eines autonomen Fahrzeugs. Die im ersten Modul 
festgestellten Schwächen wurden gezielt adressiert. Dazu zählten unter anderem 
die späte Inbetriebnahme der Elektronik, ein zu hohes Gesamtgewicht, mangelnde 
Abstimmung zwischen den Disziplinen sowie unklare Zuständigkeiten. In PREN 2 
gelang es, durch eine sprintbasierte und interdisziplinär koordinierte 
Arbeitsweise einen durchgängig lauffähigen Entwicklungsprozess zu etablieren.

Das mechanische Design wurde vollständig überarbeitet und auf additive Fertigung 
umgestellt, was signifikante Vorteile bei Gewicht, Zugänglichkeit und Änderbarkeit 
brachte. Die Elektronik konnte fristgerecht in Betrieb genommen werden, inklusive 
robuster Kommunikationsschnittstellen und Stromversorgung. Die Regelung und 
Steuerung des Fahrzeugs sowie die Sensorik zur Navigation und Hinderniserkennung 
wurden erfolgreich integriert und in umfangreichen Tests validiert. Auch die 
Umsetzung des Greifmechanismus und die Einhaltung der sicherheitsrelevanten 
Anforderungen, insbesondere der Notausfunktion, wurden zuverlässig realisiert. 
Somit erfüllt das Gesamtsystem die wichtigsten funktionalen Anforderungen und 
bildet ein solides Resultat der interdisziplinären Produktentwicklung.

\subsubsection*{Ausblick}
Auf Grundlage der erfolgreich abgeschlossenen Integration aller Subsysteme und 
der stabilen Ergebnisse aus den Feldtests kann von einer zuverlässigen 
Funktionsweise des Fahrzeugs im Wettbewerbsumfeld ausgegangen werden. Die 
Anforderungen an Navigation, Hindernisbewältigung, Zielerkennung und Reaktion 
auf dynamische Bedingungen wurden im Testbetrieb erfüllt. In diesem Zusammenhang 
erscheint eine erfolgreiche Wettbewerbsteilnahme realistisch.

Das Projekt zeigt, wie die systematische Bearbeitung technischer 
Herausforderungen, kombiniert mit enger Teamarbeit, zu einem leistungsfähigen 
Gesamtsystem führen kann. Die Erfahrungen aus PREN~2 bestätigen die Bedeutung 
frühzeitiger Systemintegration, klarer Verantwortlichkeiten und iterativer 
Verbesserungsprozesse. Auch wenn das Wettbewerbsresultat noch offen ist, ist das 
Team überzeugt, mit dem entwickelten Fahrzeug eine konkurrenzfähige Lösung 
präsentieren zu können.

\end{document}
