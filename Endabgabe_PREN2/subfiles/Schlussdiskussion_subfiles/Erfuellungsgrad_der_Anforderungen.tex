\documentclass[main.tex]{subfiles} % Subfile-Class

%==============================================================================%
%                                   Subfile                                    %
%==============================================================================%

\begin{document}

% Template

\subsection{Erfüllungsgrad der Anforderungen}

Im Rahmen der Projektumsetzung konnte der Grossteil der in der Anforderungsliste definierten Vorgaben 
erfolgreich erfüllt werden. Die zentralen funktionalen Anforderungen, wie etwa die autonome Navigation 
entlang der Leitlinien, die zuverlässige Hinderniserkennung und -bewältigung sowie das präzise 
Positionieren und Zurücksetzen von Objekten, wurden vollständig umgesetzt und anhand praktischer 
Tests nachgewiesen. Auch viele Wunschkriterien, etwa zur Nachhaltigkeit, zur Modularität der 
Konstruktion und zur Bedienbarkeit über grafische Schnittstellen, fanden Berücksichtigung und 
wurden soweit möglich realisiert.

Lediglich wenige Anforderungen, insbesondere einzelne Wunschkriterien konnten aus
zeitlichen, technischen oder budgetären Gründen nicht vollständig erfüllt werden. 
Diese Punkte beeinträchtigen jedoch weder die Funktionstüchtigkeit noch die Zielerreichung des Gesamtsystems.

Eine vollständige Übersicht über den Erfüllungsgrad sämtlicher Anforderungen ist der 
Anforderungsliste im Anhang~\ref{apx:Anforderungsliste}  zu entnehmen. Die zentralen technischen Anforderungen wurden zudem 
durch gezielte Verifikation und Tests überprüft. Eine detaillierte Darstellung dieser 
Testverfahren und deren Ergebnisse findet sich im Kapitel~\ref{sec:Testing}.

\end{document}
