% bibliography using biblatex

% --------------- in main.tex
\usepackage[backend=biber, style=numeric]{biblatex}
\addbibresource{references.bib} % Name of your .bib file

\begin{document}

\section{Introduction}
This is an example citation~\cite{einstein1905}.

\printbibliography[]

\end{document}

% --------------- in references.bib
% Book Entry
@book{knuth1997,
    author    = {Donald E. Knuth},
    title     = {The Art of Computer Programming},
    edition   = {3rd},
    publisher  = {Addison-Wesley},
    year      = {1997},
    volume    = {1},
    address   = {Reading, MA}
}

% Article in Journal
@article{einstein1905,
    author  = {Albert Einstein},
    title   = {On the Electrodynamics of Moving Bodies},
    journal = {Annalen der Physik},
    volume  = {322},
    number  = {10},
    pages   = {891--921},
    year    = {1905}
}

% Conference Paper
@inproceedings{smith2020,
    author    = {John Smith and Jane Doe},
    title     = {A New Approach to Quantum Computing},
    booktitle = {Proceedings of the 2020 International Conference on Quantum Computing},
    pages     = {101--110},
    year      = {2020},
    publisher = {IEEE},
    address   = {New York, NY}
}

% Thesis or Dissertation
@phdthesis{miller2018,
    author    = {Alice Miller},
    title     = {Machine Learning for Natural Language Processing},
    school    = {University of Example},
    year      = {2018},
    address   = {Example City, Country}
}

% Report
@techreport{johnson2019,
    author      = {Emily Johnson},
    title       = {Analysis of Recent Trends in Artificial Intelligence},
    institution = {Example Institute},
    year        = {2019},
    number      = {TR-123},
    address     = {Example City, Country}
}

% Website
@online{wikipedia2024,
    author    = {{Wikipedia Contributors}},
    title     = {OpenAI},
    year      = {2024},
    url       = {https://en.wikipedia.org/wiki/OpenAI},
    note      = {Accessed: 2024-09-16}
}

% Book Chapter
@incollection{doe2021,
    author    = {Jane Doe},
    title     = {Advanced Topics in Machine Learning},
    booktitle = {Handbook of Machine Learning},
    editor    = {John Smith and Alice Miller},
    publisher = {Academic Press},
    year      = {2021},
    pages     = {45--67},
    address   = {San Diego, CA}
}

% Miscellaneous Entry
@misc{green2022,
    author    = {Robert Green},
    title     = {Understanding the Basics of Quantum Mechanics},
    year      = {2022},
    howpublished = {Lecture Notes},
    note      = {Available at Example University}
}

Book Entry (@book): Für Bücher. Enthält Autor, Titel, Herausgeber, Jahr,
Auflage, Band und Adresse.

Article in Journal (@article): Für Zeitschriftenartikel. Enthält Autor, Titel,
Zeitschrift, Jahr, Band, Nummer und Seiten.

Conference Paper (@inproceedings): Für Konferenzbeiträge. Enthält Autor, Titel,
Konferenzname, Jahr, Seiten, Verlag und Adresse.

Thesis or Dissertation (@phdthesis): Für Dissertationen oder Masterarbeiten.
Enthält Autor, Titel, Universität, Jahr und Adresse.

Report (@techreport): Für technische Berichte. Enthält Autor, Titel,
Institution, Jahr, Berichtnummer und Adresse.

Website (@online): Für Online-Quellen. Enthält Autor (meistens keine), Titel,
Jahr, URL und Zugriffdatum.

Book Chapter (@incollection): Für Buchkapitel. Enthält Autor, Kapitel Titel,
Buch Titel, Herausgeber, Verlag, Jahr, Seiten und Adresse.

Miscellaneous Entry (@misc): Für alle anderen Quellen. Kann für verschiedene
Zwecke verwendet werden, wie z.B. Vorlesungsnotizen.