\documentclass[main.tex]{subfiles} % Subfile-Class

%==============================================================================%
%                                   Subfile                                    %
%==============================================================================%

\begin{document}

\newcounter{counter}
\setcounter{counter}{0}

\subsection{Risiken}

\subsubsection{Maschinenbau}
\begin{table}[H]

    \begin{tabularx}{\textwidth}{|>{\centering\arraybackslash}p{0.5cm}|>{\raggedright\arraybackslash}p{1.5cm}|>{\raggedright\arraybackslash}X|>{\centering\arraybackslash}p{0.75cm}|>{\centering\arraybackslash}p{0.75cm}|>{\raggedright\arraybackslash}X|}
        \hline
        \textbf{\#}                                 & \textbf{Sprint} & \textbf{Risiko} & \textbf{SA} & \textbf{EW} & \textbf{Auswirkungen} \\

        \hline
        % ---------------------------------------------------------------------------------------------------------------------------------------------------------
        \rowcolor{white!30}
        \refstepcounter{counter} 1.\arabic{counter} & Prepare         & Fehldruck vom 3D Drucker      & 2            & 2             & Vorbereitungsphase verzögert sich       \\
        \hline

        % ---------------------------------------------------------------------------------------------------------------------------------------------------------
        \rowcolor{white!30}
        \refstepcounter{counter} 1.\arabic{counter} & Prepare         & Beim Zusammenbau des Roboters bricht ein Teil       & 3             & 3             & Vorbereitungsphase verzögert sich     \\
        \hline

        % ---------------------------------------------------------------------------------------------------------------------------------------------------------
        \rowcolor{white!30}
        \refstepcounter{counter} 1.\arabic{counter} &                 &                 &             &             &                       \\
        \hline


    \end{tabularx}
    \caption{Erkannte Risiken aus dem Bereich der Mechanik}
\end{table}

\subsubsection{Elektrotechnik}
\setcounter{counter}{0}
\begin{table}[H]
    \begin{tabularx}{\textwidth}{|>{\centering\arraybackslash}p{0.5cm}|>{\raggedright\arraybackslash}p{1.5cm}|>{\raggedright\arraybackslash}X|>{\centering\arraybackslash}p{0.75cm}|>{\centering\arraybackslash}p{0.75cm}|>{\raggedright\arraybackslash}X|}
        \hline
        \textbf{\#}                                 & \textbf{Sprint} & \textbf{Risiko}                                                         & \textbf{SA} & \textbf{EW} & \textbf{Auswirkungen}                                                                                         \\

        \hline
        % ---------------------------------------------------------------------------------------------------------------------------------------------------------
        \rowcolor{white!30}
        \refstepcounter{counter} 2.\arabic{counter} & Prepare         & Beim Zusammenlöten der Prints kam es zu Fehlern                         & 4           & 3           & Vorbereitungsphase verzögert sich                                                                             \\
        \hline
        % ---------------------------------------------------------------------------------------------------------------------------------------------------------
        \rowcolor{white!30}
        \refstepcounter{counter} 2.\arabic{counter} & 2               & Powerbudget ist nicht ausreichend                                       & 3           & 2           & Es müssen Einsparungen bei der Leistung der Motoren getroffen werden                                          \\
        \hline
        % ---------------------------------------------------------------------------------------------------------------------------------------------------------
        \rowcolor{white!30}
        \refstepcounter{counter} 2.\arabic{counter} & 1               & Bei Printplattenentwicklung kam es zu Fehlern                           & 4           & 2           & Das Konzept muss umgestellt werden, Aufgaben z.B. von anderen PCB's übernommen.                               \\
        \hline
        % ---------------------------------------------------------------------------------------------------------------------------------------------------------
        \rowcolor{white!30}
        \refstepcounter{counter} 2.\arabic{counter} & 1               & RS422 Schnittstelle und das Busprotokoll funktioniert nicht zuverlässig & 4           & 2           & Es sind Steckplätze zum Überbrücken dieser Schnittstelle vorgesehen, darauf muss dann zurückgegriffen werden. \\
        \hline

    \end{tabularx}
    \caption{Erkannte Risiken aus dem Bereich der Elektrotechnik}
\end{table}

\subsubsection{Informatik}
\setcounter{counter}{0}
\begin{table}[H]
    \begin{tabularx}{\textwidth}{|>{\centering\arraybackslash}p{0.5cm}|>{\raggedright\arraybackslash}p{1.5cm}|>{\raggedright\arraybackslash}X|>{\centering\arraybackslash}p{0.75cm}|>{\centering\arraybackslash}p{0.75cm}|>{\raggedright\arraybackslash}X|}
        \hline
        \textbf{\#}                                 & \textbf{Sprint} & \textbf{Risiko} & \textbf{SA} & \textbf{EW} & \textbf{Auswirkungen} \\

        \hline
        % ---------------------------------------------------------------------------------------------------------------------------------------------------------
        \rowcolor{white!30}
        \refstepcounter{counter} 3.\arabic{counter} &                 &                 &             &             &                       \\
        \hline

    \end{tabularx}
    \caption{Erkannte Risiken aus dem Bereich der Informatik}
\end{table}

\end{document}