\documentclass[main.tex]{subfiles} % Subfile-Class

%==============================================================================%
%                                   Subfile                                    %
%==============================================================================%

\begin{document}

\newcounter{counter}
\setcounter{counter}{0}

\subsection{Risiken}

Im nachfolgenden Abschnitt werden verschiedene Risiken aufgeführt, die während
der Projektlaufzeit auftreten können. Die Risiken sind in die drei Bereiche
Maschinenbau, Elektrotechnik und Informatik unterteilt. Ergänzend dazu wird
aufgeführt, in welchem Sprint das Risiko relevant wird, wie die
Eintrittswahrscheinlichkeit sowie das Schadensausmass aussehen. Diese Risiken
werden zum Start eines jeden Sprints neu bewertet und bei der Aufgabenplanung
mit einer höheren Priorität versehen.

Eine Auflistung der zu den Risiken ergreifenden Massnahmen ist daher noch nicht
Teil der vorliegenden Abgabe, da diese erst im Verlauf des Projektes erarbeitet
werden.Das Projektteam ist sich der Risiken bewusst und kann so priorisiert und
sensibilisiert an die Aufgaben herangehen.

\subsubsection{Maschinenbau}
\begin{table}[H]

    \begin{tabularx}{\textwidth}{|>{\centering\arraybackslash}p{0.5cm}|>{\raggedright\arraybackslash}p{1.5cm}|>{\raggedright\arraybackslash}X|>{\centering\arraybackslash}p{0.75cm}|>{\centering\arraybackslash}p{0.75cm}|>{\raggedright\arraybackslash}X|}
        \hline
        \textbf{\#}                                 & \textbf{Sprint} & \textbf{Risiko}                                                    & \textbf{SA} & \textbf{EW} & \textbf{Auswirkungen}                                                               \\

        \hline
        % ---------------------------------------------------------------------------------------------------------------------------------------------------------
        \rowcolor{white!30}
        \refstepcounter{counter} 1.\arabic{counter} & Prepare         & Fehldruck vom 3D Drucker                                           & 2           & 2           & Vorbereitungsphase verzögert sich                                                   \\
        \hline

        % ---------------------------------------------------------------------------------------------------------------------------------------------------------
        \rowcolor{white!30}
        \refstepcounter{counter} 1.\arabic{counter} & Prepare         & Beim Zusammenbau des Roboters bricht ein Teil                      & 3           & 3           & Vorbereitungsphase verzögert sich                                                   \\
        \hline
        % ---------------------------------------------------------------------------------------------------------------------------------------------------------
        \rowcolor{white!30}
        \refstepcounter{counter} 1.\arabic{counter} & Prepare         & Laufkugel bleibt hängen in Fuge                                    & 2           & 2           & Es muss ein neuer Adapter angefertigt werden für eine grössere Laufkugel.           \\
        \hline
        % ---------------------------------------------------------------------------------------------------------------------------------------------------------
        \rowcolor{white!30}
        \refstepcounter{counter} 1.\arabic{counter} & 1               & Durchdrehen der Räder bei Beschleunigung                           & 1           & 3           & Fahrgeschwindigkeit muss reduziert werden.                                          \\
        \hline
        % ---------------------------------------------------------------------------------------------------------------------------------------------------------
        \rowcolor{white!30}
        \refstepcounter{counter} 1.\arabic{counter} & 1               & Kamerapositionierung nicht stabil genug für zuverlässige Aufnahmen & 3           & 3           & Anpassung der Kamera-Halterung notwendig.                                           \\
        \hline
        % ---------------------------------------------------------------------------------------------------------------------------------------------------------
        \rowcolor{white!30}
        \refstepcounter{counter} 1.\arabic{counter} & 2               & Maximalgewicht wird nicht eingehalten                              & 4           & 2           & Chassiskonzept muss nochmals überarbeitet werden, um noch mehr Gewicht einzusparen. \\
        \hline

    \end{tabularx}
    \caption{Erkannte Risiken aus dem Bereich der Mechanik}
\end{table}

\subsubsection{Elektrotechnik}
\setcounter{counter}{0}
\begin{table}[H]
    \begin{tabularx}{\textwidth}{|>{\centering\arraybackslash}p{0.5cm}|>{\raggedright\arraybackslash}p{1.5cm}|>{\raggedright\arraybackslash}X|>{\centering\arraybackslash}p{0.75cm}|>{\centering\arraybackslash}p{0.75cm}|>{\raggedright\arraybackslash}X|}
        \hline
        \textbf{\#}                                 & \textbf{Sprint} & \textbf{Risiko}                                                         & \textbf{SA} & \textbf{EW} & \textbf{Auswirkungen}                                                                                         \\

        \hline
        % ---------------------------------------------------------------------------------------------------------------------------------------------------------
        \rowcolor{white!30}
        \refstepcounter{counter} 2.\arabic{counter} & Prepare         & Beim Zusammenlöten der Prints kam es zu Fehlern                         & 4           & 3           & Vorbereitungsphase verzögert sich                                                                             \\
        \hline
        % ---------------------------------------------------------------------------------------------------------------------------------------------------------
        \rowcolor{white!30}
        \refstepcounter{counter} 2.\arabic{counter} & 2               & Powerbudget ist nicht ausreichend                                       & 3           & 2           & Es müssen Einsparungen bei der Leistung der Motoren getroffen werden                                          \\
        \hline
        % ---------------------------------------------------------------------------------------------------------------------------------------------------------
        \rowcolor{white!30}
        \refstepcounter{counter} 2.\arabic{counter} & 1               & Bei Printplattenentwicklung kam es zu Fehlern                           & 4           & 2           & Das Konzept muss umgestellt werden, Aufgaben z.B. von anderen PCB's übernommen.                               \\
        \hline
        % ---------------------------------------------------------------------------------------------------------------------------------------------------------
        \rowcolor{white!30}
        \refstepcounter{counter} 2.\arabic{counter} & 1               & RS422 Schnittstelle und das Busprotokoll funktioniert nicht zuverlässig & 4           & 2           & Es sind Steckplätze zum Überbrücken dieser Schnittstelle vorgesehen, darauf muss dann zurückgegriffen werden. \\
        \hline

    \end{tabularx}
    \caption{Erkannte Risiken aus dem Bereich der Elektrotechnik}
\end{table}

\subsubsection{Informatik}
\setcounter{counter}{0}
\begin{table}[H]
    \begin{tabularx}{\textwidth}{|>{\centering\arraybackslash}p{0.5cm}|>{\raggedright\arraybackslash}p{1.5cm}|>{\raggedright\arraybackslash}X|>{\centering\arraybackslash}p{0.75cm}|>{\centering\arraybackslash}p{0.75cm}|>{\raggedright\arraybackslash}X|}
        \hline
        \textbf{\#}                                 & \textbf{Sprint} & \textbf{Risiko}                                                  & \textbf{SA} & \textbf{EW} & \textbf{Auswirkungen}                                                                                              \\
        \hline
        % ---------------------------------------------------------------------------------------------------------------------------------------------------------
        \rowcolor{white!30}
        \refstepcounter{counter} 3.\arabic{counter} & Prepare         & Kommunikation zwischen Controllern funktioniert nicht            & 3           & 2           & Vorbereitungsphase verzögert sich, ggf. muss alternative Kommunikation implementiert werden.                       \\
        \hline
        % ---------------------------------------------------------------------------------------------------------------------------------------------------------
        \rowcolor{white!30}
        \refstepcounter{counter} 3.\arabic{counter} & 1               & Bildverarbeitung (Interpretation) funktioniert nicht zuverlässig & 4           & 2           & Algorithmus muss angepasst werden, was zu Verzögerungen führen kann.                                               \\
        \hline
        % ---------------------------------------------------------------------------------------------------------------------------------------------------------
        \rowcolor{white!30}
        \refstepcounter{counter} 3.\arabic{counter} & 1               & Prozessorleistung für die Bildverarbeitung ist zu gering         & 4           & 2           & Leistungsoptimierung (andere Programmiersprache) oder Hardware-Upgrade erforderlich, Zeitplan kann sich verzögern. \\
        \hline
    \end{tabularx}
    \caption{Erkannte Risiken aus dem Bereich der Informatik}
\end{table}

\end{document}
